% The contents of this file is 
% Copyright (c) 2009-  Charles R. Severance, All Righs Reserved

% LATEXONLY

\input{latexonly}

\newtheorem{ex}{Exercise}[chapter]

\begin{latexonly}

\renewcommand{\blankpage}{\thispagestyle{empty} \quad \newpage}

\thispagestyle{empty}

\begin{flushright}
\vspace*{2.0in}

\begin{spacing}{3}
{\huge 정보교육을 위한 파이썬}\\
{\Large 정보 탐색}
\end{spacing}

\vspace{0.25in}

Version \theversion

\vspace{0.5in}


{\Large
저자: Charles Severance\\
번역: 이광춘, 한정수 \\
(xwmooc)
}

\vfill

\end{flushright}

%--copyright--------------------------------------------------
\pagebreak
\thispagestyle{empty}

{\small
Copyright \copyright ~2009- Charles Severance.


출판 이력:

\begin{description}

\item[2014년 9월:] xwmooc 프로젝트 일환으로 \emph{''정보과학을 위한 파이썬''} 으로 제목 정하고 한국어로 번역 공개

\item[2013년 10월:] JSON로 전환, OAuth 사용. 13장, 14장에 주요 개정
신규 시각화 장 추가.

\item[2013년 9월:] Amazon CreateSpace 책 출판

\item[2010년 1월 :] 미시건 대학 Espresso Book Machine 사용 책 출판

\item[2009년 12월:] \emph{Think Python: How to Think Like a Computer Scientist}에서 2장 {\verb"~"} 10장까지 주요 개정 \\
\emph{Python for Informatics: Exploring Information}을 위해 1장, 11장 {\verb"~"} 15장 저작

\item[2008년 6월:] \emph{Think Python: How to Think Like a Computer Scientist} 제목 바꾸고, 주요 개정.

\item[2007년 8월:] \emph{How to Think Like a (Python) Programmer} 제목 바꾸고, 주요 개정.

\item[2002년 4월:] \emph{How to Think Like a Computer Scientist} 초판 공개

\end{description}

\vspace{0.2in}

이 책은 크리에이티브 커먼즈 라이선스 3.0 (Creative Commons Attribution-NonCommercial-Share Alike 3.0)으로 인가되었다.
라이선스의 자세한 사항은 \url{creativecommons.org/licenses/by-nc-sa/3.0/}에 기재되어 있다.
저작권 상세 부록에서 저자가 생각하는 상업적, 비상업적 이용 그리고 라이센스 면제를 생각하는 바를 확인할 수 있다.
이책의 \emph{Think Python: How to Think Like a Computer Scientist}의 \LaTeX\ 소스는 
\url{http://www.thinkpython.com}에서 이용가능하다.

\vspace{0.2in}

} % end small

\end{latexonly}


% HTMLONLY

\begin{htmlonly}

% TITLE PAGE FOR HTML VERSION

{\Large \thetitle}

{\large 
Charles Severance}

Version \theversion

\setcounter{chapter}{-1}

\end{htmlonly}
