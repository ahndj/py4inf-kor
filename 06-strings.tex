% LaTeX source for ``Python for Informatics: Exploring Information''
% Copyright (c)  2010-  Charles R. Severance, All Rights Reserved
% 한국어 번역 : 이광춘, 한정수

\chapter{문자열}
\label{strings}


\section{    문자열은 순서(sequence)다.}
\index{순서 (sequence)}
\index{문자 (character)}
\index{꺾쇠 연산자 (bracket operator)}
\index{연산자 (operator)!bracket}

문자열은 여러 문자들의 순서다. 
꺾쇠 연산자로 한번에 하나씩 문자에 접근한다.

\beforeverb
\begin{verbatim}
>>> fruit = 'banana'
>>> letter = fruit[1]
\end{verbatim}
\afterverb
%
\index{인덱스 (index)}

두 번째 문장은 변수 {\tt fruit}에서 1번 위치 문자를 추출하여 변수 {\tt letter}에 대입한다.
꺾쇠 표현식을 {\bf 인덱스(index)}라고 부른다. 
인덱스는 순서(sequence)에서 사용자가 어떤 문자를 원하는지 표시한다.

하지만, 여러분이 기대한 것은 아니다.

\beforeverb
\begin{verbatim}
>>> print letter
a
\end{verbatim}
\afterverb
%

대부분의 사람에게 \verb"'banana'"의 첫 분자는 {\tt a}가 아니라 {\tt b}다.
하지만, 파이썬 인텍스는 문자열 처음부터 오프셋(offset)\footnote{컴퓨터에서 어떤 주소로부터 간격을 두고 떨어진 주소와의 거리. 기억 장치가 페이지 혹은 세그먼트 단위로 나누어져 있을 때 하나의 시작 주소로부터 오프셋만큼 떨어진 위치를 나타내는 것이다. [네이버 지식백과] 오프셋 [offset] (IT용어사전, 한국정보통신기술협회)} 이다. 
첫 글자 오프셋은 0 이다.

\beforeverb
\begin{verbatim}
>>> letter = fruit[0]
>>> print letter
b
\end{verbatim}
\afterverb
%

그래서, {\tt b}가 \verb"'banana'"의 0 번째 문자가 되고 {\tt a}가 첫번째, {\tt n}이 두번째 문자가 된다.

\beforefig
\centerline{\includegraphics[height=0.50in]{figs2/string.eps}}
\afterfig

\index{인덱스 (index)!0 에서 시작(starting at zero)}
\index{0 (zero), 인덱스 시작점 (index starting at)}

인덱스로 문자와 연산자를 포함하는 어떤 표현식도 사용가능지만, 인덱스 값은 정수여야만 한다.
정수가 아닌 경우 다음과 같은 결과를 얻게 된다.

\index{인덱스 (index)}
\index{예외 (exception)!자료형 오류 (TypeError)}
\index{자료형 오류 (TypeError)}

\beforeverb
\begin{verbatim}
>>> letter = fruit[1.5]
TypeError: string indices must be integers
\end{verbatim}
\afterverb
%

\section{    {\tt len}함수 사용 문자열 길이 구하기}

\index{len 함수 (len function)}
\index{함수 (function)!len}

{\tt len} 함수는 문자열의 문자 갯수를 반환하는 내장함수다.

\beforeverb
\begin{verbatim}
>>> fruit = 'banana'
>>> len(fruit)
6
\end{verbatim}
\afterverb
%

문자열의 가장 마지막 문자를 얻기 위해서, 아래와 같이 시도하려 싶을 것이다.

\index{예외 (exception)!인덱스 오류 (IndexError)}
\index{인덱스 오류 (IndexError)}

\beforeverb
\begin{verbatim}
>>> length = len(fruit)
>>> last = fruit[length]
IndexError: string index out of range
\end{verbatim}
\afterverb
%

{\tt 인덱스 오류 (IndexError)} 이유는 {\tt 'banana'} 에 6번 인텍스 문자가 없기 때문이다.
0 에서부터 시작했기 때문에 6개 문자는 0 에서부터 5 까지 번호가 매겨졌다. 
마지막 문자를 얻기 위해서 {\tt length}에서 1을 빼야 한다.

\beforeverb
\begin{verbatim}
>>> last = fruit[length-1]
>>> print last
a
\end{verbatim}
\afterverb
%

대안으로 음의 인텍스를 사용해서 문자열 끝에서 역으로 수를 셀 수 있다.
표현식 {\tt fruit[-1]}은 마지막 문자를 {\tt fruit[-2]}는 끝에서 두 번째 등등 활용할 수 있다.

\index{인덱스 (index)!음수 (negative)}
\index{음수 인덱스 (negative index)}

\section{    루프를 사용한 문자열 운행법}
\label{for}

\index{운행법 (traversal)}
\index{loop!traversal}
\index{for 문 (for loop)}
\index{루프 (loop)!for}
\index{문장 (statement)!for}

연산의 많은 경우에 문자열을 한번에 한 문자씩 처리한다. 
종종 처음에서 시작해서, 차례로 각 문자를 선택하고, 선택된 문자에 임의 연산을 수행하고, 끝까지 계속한다. 
이런 처리 패턴을 {\bf 운행법(traversal)}라고 한다.
운행법을 작성하는 한 방법이 {\tt while} 루프다.

\beforeverb
\begin{verbatim}
index = 0
while index < len(fruit):
    letter = fruit[index]
    print letter
    index = index + 1
\end{verbatim}
\afterverb
%

{\tt while} 루프가 문자열을 운행하여 문자열을 한줄에 한 글자씩 화면에 출력한다.
루프 조건이 {\tt index < len(fruit)}이여서, {\tt index}가 문자열 길이와 같을 때,
조건은 거짓이 되고, 루프의 몸통 부문은 실행이 되지 않는다. 
파이썬이 접근한 마지막 {\tt len(fruit)-1} 인텍스 문자로, 문자열의 마지막 문자다.

\begin{ex}
문자열의 마지막 문자에서 시작해서, 문자열 처음으로 역진행하면서 한줄에 한자씩 화면에 출력하는 {\tt while} 루프를 작성하세요.
\end{ex}

운행법을 작성하는 또 다른 방법은 {\tt for} 루프다.

\beforeverb
\begin{verbatim}
for char in fruit:
    print char
\end{verbatim}
\afterverb
%

루프를 매번 반복할 때, 문자열 다음 문자가 변수 {\tt char}에 대입된다. 
루프는 더 이상 남겨진 문자가 없을 때까지 계속 실행된다.

\section{    문자열 슬라이스(slice)}
\label{slice}

\index{슬라이스 연산자 (slice operator)}
\index{연산자 (operator)!슬라이스 (slice)}
\index{인덱스 (index)!슬라이스 (slice)}
\index{문자열 (string)!슬라이스 (slice)}
\index{슬라이스 (slice)!문자열 (string)}

문자열의 일부분을 {\bf 슬라이스(slice)}라고 한다. 
문자열 슬라이스를 선택하는 것은 문자를 선택하는 것과 유사하다.

\beforeverb
\begin{verbatim}
>>> s = 'Monty Python'
>>> print s[0:5]
Monty
>>> print s[6:13]
Python
\end{verbatim}
\afterverb
%

{\tt [n:m]} 연산자는 n번째 문자부터 m번째 문자까지의 문자열 - 첫 번째는 포함하지만 마지막은 제외 - 부분을 반환한다.

콜론 앞 첫 인텍스를 생략하면, 문자열 슬라이스는 문자열 처음부터 시작한다.
두 번째 인텍스를 생략하면, 문자열 슬라이스는 문자열 끝까지 간다. 

\beforeverb
\begin{verbatim}
>>> fruit = 'banana'
>>> fruit[:3]
'ban'
>>> fruit[3:]
'ana'
\end{verbatim}
\afterverb
%

만약 첫번째 인텍스가 두번째보다 크거나 같은 경우 결과는 인용부호로 표현되는 {\bf 빈 문자열(empty string)}이 된다.

\index{인용 부호 (quotation mark)}

\beforeverb
\begin{verbatim}
>>> fruit = 'banana'
>>> fruit[3:3]
''
\end{verbatim}
\afterverb
%

빈 문자열은 어떤 문자도 포함하지 않아서 길이가 0 이 되지만, 이것을 제외하고 다른 문자열과 동일하다.

\begin{ex}
{\tt fruit}이 문자열로 주어졌을 때, {\tt fruit[:]}의 의미는 무엇인가요?

\index{복사 (copy)!슬라이스 (slice)}
\index{슬라이스 (slice)!복사 (copy)}

\end{ex}


\section{    문자열은 불변이다.}
\index{가변성 (mutability)}
\index{불변성 (immutability)}
\index{문자열 (string)!불변성 (immutable)}

문자열 내부에 있는 문자를 변경하려고 대입문 왼쪽편에 {\tt []} 연산자를 사용하고 싶은 유혹이 있을 것이다.
예를 들어 다음과 같다.

\index{자료형 오류 (TypeError)}
\index{예외 (exception)!자료형 오류 (TypeError)}

\beforeverb
\begin{verbatim}
>>> greeting = 'Hello, world!'
>>> greeting[0] = 'J'
TypeError: object does not support item assignment
\end{verbatim}
\afterverb
%

이 경우 ''객체(object)''는 문자열이고, 대입하고자 하는 문자는 ''항목(item)''이다. 
지금으로서 {\bf 객체}는 값과 동일하지만, 나중에 객체 정의를 좀더 상세화할 것이다.
{\bf 항목}은 순서 값 중의 하나다.

\index{객체 (object)}
\index{항목 대입 (item assignment)}
\index{대입 (assignment)!항목 (item)}
\index{불변성 (immutability)}

오류 이유는 문자열은 {\bf 불변(immutable)}이기 때문이다. 
따라서 기존 문자열을 변경할 수 없다는 의미다.
최선의 방법은 원래 문자열을 변형한 새로운 문자열을 생성하는 것이다.

\beforeverb
\begin{verbatim}
>>> greeting = 'Hello, world!'
>>> new_greeting = 'J' + greeting[1:]
>>> print new_greeting
Jello, world!
\end{verbatim}
\afterverb
%

새로운 첫 문자에 {\tt greeting} 문자열 슬라이스를 연결한다. 
원래 문자열에는 어떤 영향도 주지 않는 새로운 문자열을 생성되었다.

\index{접합 (concatenation)}

\section{    루프 돌기(looping) 계수(counting)}
\label{counter}

\index{계수기 (counter)}
\index{계수와 루프 돌기 (counting and looping)}
\index{루프 돌기와 계수 (looping and counting)}
\index{루프 돌기 (looping)!문자열 가지고 (with strings)}

다음 프로그램은 문자열에 문자 {\tt a}가 나타나는 횟수를 계수한다.

\beforeverb
\begin{verbatim}
word = 'banana'
count = 0
for letter in word:
    if letter == 'a':
        count = count + 1
print count
\end{verbatim}
\afterverb
%

상기 프로그램은 {\bf 계수기(counter)}라고 부르는 또다른 연산 패턴을 보여준다. 
변수 {\tt count}는 0 으로 초기화 되고, 매번 {\tt a}를 찾을 때마다 증가한다.
루프를 빠져나갔을 때, {\tt count}는 결과 값 즉, {\tt a}가 나타난 총 횟수를 담고 있다.

\begin{ex}
\index{캡슐화 (encapsulation)}

문자열과 문자를 인자(argument)로 받도록 상기 코드를 {\tt count}라는 함수로 캡슐화(encapsulation)하고 일반화하세요.

\end{ex}

\section{    {\tt in} 연산자}
\label{inboth}

\index{in 연산자 (in operator)}
\index{연산자 (operator)!in}
\index{불 연산자 (boolean operator)}
\index{연산자 (operator)!불 (boolean)}

연산자 {\tt in} 은 불 연산자로 두 개의 문자열을 받아, 첫 번째 문자열이 두 번째 문자열의 일부이면 {\tt 참(True)}을 반환한다.

\beforeverb
\begin{verbatim}
>>> 'a' in 'banana'
True
>>> 'seed' in 'banana'
False
\end{verbatim}
\afterverb
%

\section{    문자열 비교}

\index{문자열 (string)!비교 (comparison)}
\index{비교 (comparison)!문자열 (string)}

비교 연산자도 문자열에서 동작한다. 
두 문자열이 같은지를 살펴보다.

\beforeverb
\begin{verbatim}
if word == 'banana':
    print  'All right, bananas.'
\end{verbatim}
\afterverb
%

다른 비교 연산자는 단어를 알파벳 순서로 정렬하는데 유용하다.

\beforeverb
\begin{verbatim}
if word < 'banana':
    print 'Your word,' + word + ', comes before banana.'
elif word > 'banana':
    print 'Your word,' + word + ', comes after banana.'
else:
    print 'All right, bananas.'
\end{verbatim}
\afterverb
%

파이썬은 사람과 동일하는 방식으로 대문자와 소문자를 다루지 않는다.
모든 대문자는 소문자 앞에 온다.

\beforeverb
\begin{verbatim}
Your word, Pineapple, comes before banana.
\end{verbatim}
\afterverb
%

이러한 문제를 다루는 일반적인 방식은 비교 연산을 수행하기 전에 문자열을 표준 포맷으로 예를 들어 모두 소문자, 변환하는 것입니다.
경우에 따라서 ''Pineapple''로 무장한 사람들로부터 여러분을 보호해야하는 것을 명심하세요.

\section{    {\tt string} 메쏘드}
문자열은 파이썬 {\bf 객체(objects)}의 한 예다. 
객체는 데이터(실제 문자열 자체)와 {\bf 메쏘드(methods)}를 담고 있다.
메쏘드는 객체에 내장되고 어떤 객체의 {\bf 인스턴스(instance)}에도 사용되는 사실상 함수다.

객체에 대해 이용가능한 메쏘드를 보여주는 {\tt dir} 함수가 파이썬에 있다.
{\tt type} 함수는 객체의 자료형(type)을 보여 주고, {\tt dir}은 객체에 사용될 수 있는 메쏘드를 보여준다.

\beforeverb
\begin{verbatim}
>>> stuff = 'Hello world'
>>> type(stuff)
<type 'str'>
>>> dir(stuff)
['capitalize', 'center', 'count', 'decode', 'encode', 
'endswith', 'expandtabs', 'find', 'format', 'index', 
'isalnum', 'isalpha', 'isdigit', 'islower', 'isspace', 
'istitle', 'isupper', 'join', 'ljust', 'lower', 'lstrip', 
'partition', 'replace', 'rfind', 'rindex', 'rjust', 
'rpartition', 'rsplit', 'rstrip', 'split', 'splitlines', 
'startswith', 'strip', 'swapcase', 'title', 'translate', 
'upper', 'zfill']
>>> help(str.capitalize)
Help on method_descriptor:

capitalize(...)
    S.capitalize() -> string
    
    Return a copy of the string S with only its first character
    capitalized.
>>>
\end{verbatim}
\afterverb
%

{\tt dir} 함수가 메쏘드 목록을 보여주고, 메쏘드에 대한 간단한 문서 정보는 {\tt help}를 사용할 수 있지만,
문자열 메쏘드에 대한 좀더 좋은 문서 정보는 \url{docs.python.org/library/string.html}에서 찾을 수 있다.

인자를 받고 값을 반환한다는 점에서 {\bf 메쏘드(method)}를 호출하는 것은 함수를 호출하는 것과 유사하지만, 구문은 다르다.
구분자로 점을 사용해서 변수명에 메쏘드명을 붙여 메쏘드를 호출한다.

예를 들어, {\tt upper} 메쏘드는 문자열을 받아 모두 대문자로 변환된 새로운 문자열을 반환한다.

\index{메쏘드 (method)}
\index{문자열 (string)!메쏘드 (method)}

함수 구문 {\tt upper(word)} 대신에, {\tt word.upper()} 메쏘드 구문을 사용한다.

\index{점 표기법 (dot notation)}

\beforeverb
\begin{verbatim}
>>> word = 'banana'
>>> new_word = word.upper()
>>> print new_word
BANANA
\end{verbatim}
\afterverb
%

이런 형태의 점 표기법은 메쏘드 이름({\tt upper})과 메쏘드가 적용되는 문자열 이름({\tt word})을 명세한다.
빈 괄호는 메쏘드가 인자가 없다는 것을 나타낸다.

\index{괄호 (parentheses)!빈 (empty)}

메쏘드를 부르는 것을 {\bf 호출(invocation)}이라고 부른다. 
상기의 경우, {\tt word}에 {\tt upper} 메쏘드를 호출한다고 말한다.

\index{호출 (invocation)}

예를 들어, 문자열안에 문자열의 위치를 찾는 {\tt find}라는 문자열 메쏘드가 있다.

\beforeverb
\begin{verbatim}
>>> word = 'banana'
>>> index = word.find('a')
>>> print index
1
\end{verbatim}
\afterverb
%

상기 예제에서, {\tt word} 문자열의 {\tt find} 메쏘드를 호출하여 매개 변수로 찾고자 하는 문자를 넘긴다.

{\tt find} 메쏘드는 문자뿐만 아니라 부속 문자열(substring)도 찾을 수 있다.

\beforeverb
\begin{verbatim}
>>> word.find('na')
2
\end{verbatim}
\afterverb
%

두 번째 인자로 어디서 검색을 시작할지 인텍스를 넣을 수 있다.

\index{옵션 인자 (optional argument)}
\index{인자 (argument)!옵션 (optional)}

\beforeverb
\begin{verbatim}
>>> word.find('na', 3)
4
\end{verbatim}
\afterverb
%

한 가지 자주 있는 작업은 {\tt strip} 메쏘드를 사용해서 문자열 시작과 끝의 공백(공백 여러개, 탭, 새줄)을 제거하는 것이다.

\beforeverb
\begin{verbatim}
>>> line = '  Here we go  '
>>> line.strip()
'Here we go'
\end{verbatim}
\afterverb
%

{\bf startswith} 메쏘드는 참, 거짓 같은 불 값(boolean value)을 반환한다.

\beforeverb
\begin{verbatim}
>>> line = 'Please have a nice day'
>>> line.startswith('Please')
True
>>> line.startswith('p')
False
\end{verbatim}
\afterverb
%

{\bf startswith}가 대소문자를 구별하는 것을 요구하기 때문에 {\tt lower} 메쏘드를 사용해서 검증을 수행하기 전에, 
한 줄을 입력받아 모두 소문자로 변환하는 것이 필요하다.

\beforeverb
\begin{verbatim}
>>> line = 'Please have a nice day'
>>> line.startswith('p')
False
>>> line.lower()
'please have a nice day'
>>> line.lower().startswith('p')
True
\end{verbatim}
\afterverb
%

마지막 예제에서 문자열이 문자 ''p''로 시작하는지를 검증하기 위해서, 
{\tt lower} 메쏘드를 호출하고 나서 바로 {\tt startswith} 메쏘드를 사용한다.
주의깊게 순서만 다룬다면, 한 줄에 다수 메쏘드를 호출할 수 있다.

\begin{ex}
\index{계수 메쏘드 (count method)}
\index{메쏘드 (method)!계수 (count)}

앞선 예제와 유사한 함수인 {\tt count}로 불리는 문자열 메쏘드가 있다.
\url{docs.python.org/library/string.html}에서 {\tt count} 메쏘드에 대한 문서를 읽고,
문자열 \verb"'banana'"의 문자가 몇 개인지 계수하는 메쏘드 호출 프로그램을 작성하세요.

\end{ex}

\section{    문자열 파싱(Parsing)}

종종, 문자열을 들여다 보고 특정 부속 문자열(substring)을 찾고 싶다. 
예를 들어, 아래와 같은 형식으로 작성된 일련의 라인이 주어졌다고 가정하면,

\beforeverb
\begin{alltt}
From stephen.marquard@{\bf uct.ac.za} Sat Jan  5 09:14:16 2008
\end{alltt}
\afterverb

각 라인마다 뒤쪽 전자우편 주소(즉, {\tt uct.ac.za})만 뽑아내고 싶을 것이다.
{\tt find} 메쏘드와 문자열 슬라이싱(string sliceing)을 사용해서 작업을 수행할 수 있다.

우선, 문자열에서 골뱅이(@, at-sign) 기호의 위치를 찾는다. 
그리고, 골뱅이 기호 \emph{뒤} 첫 공백 위치를 찾는다. 
그리고 나서, 찾고자 하는 부속 문자열을 뽑아내기 위해서 문자열 슬라이싱을 사용한다.

\beforeverb
\begin{verbatim}
>>> data = 'From stephen.marquard@uct.ac.za Sat Jan  5 09:14:16 2008'
>>> atpos = data.find('@')
>>> print atpos
21
>>> sppos = data.find(' ',atpos)
>>> print sppos
31
>>> host = data[atpos+1:sppos]
>>> print host
uct.ac.za
>>> 
\end{verbatim}
\afterverb
%

{\tt find} 메쏘드를 사용해서 찾고자 하는 문자열의 시작 위치를 명세한다. 
문자열 슬라이싱(slicing)할 때, 골뱅기 기호 뒤부터 빈 공백을 \emph{포함하지 않는} 위치까지 문자열을 뽑아낸다.

{\tt find} 메쏘드에 대한 문서는 \url{docs.python.org/library/string.html}에서 참조 가능하다.

\section{    서식 연산자}

\index{서식 연산자 (format operator)}
\index{연산자 (operator)!서식 (format)}

{\bf 서식 연산자(format operator)}, {\tt \%}는 문자열 일부를 변수에 저장된 값으로 바꿔 문자열을 구성한다.
정수에 서식 연산자가 적용될 때, {\tt \%}는 나머지 연산자가 된다. 
하지만 첫 피연산자가 문자열이면, {\tt \%}은 서식 연산자가 된다.

\index{서식 문자열 (format string)}

첫 피연산자는 {\bf 서식 문자열 format string}로 두번째 피연산자가 어떤 형식으로 표현되는지를 명세하는 하나 혹은 그 이상의 
{\bf 서식 순서 format sequence}를 담고 있다. 
결과값은 문자열이다.

\index{서식 순서 (format sequence)}

예를 들어, 형식 순서 \verb"'%d'"의 의미는 두번째 피연산자가 정수 형식으로 표현됨을 뜻한다. ({\tt d}는 ``decimal''을 나타낸다.)

\beforeverb
\begin{verbatim}
>>> camels = 42
>>> '%d' % camels
'42'
\end{verbatim}
\afterverb
%

결과는 문자열 \verb"'42'"로 정수 {\tt 42}와 혼동하면 안 된다.

서식 순서는 문자열 어디에도 나타날 수 있어서 문장 중간에 값을 임베드(embed)할 수 있다.

\beforeverb
\begin{verbatim}
>>> camels = 42
>>> 'I have spotted %d camels.' % camels
'I have spotted 42 camels.'
\end{verbatim}
\afterverb
%

만약 문자열 서식 순서가 하나 이상이라면, 두번째 인자는 튜플(tuple)이 된다.
서식 순서 각각은 순서대로 튜플 요소와 매칭된다.

다음 예제는 정수 형식을 표현하기 위해서 \verb"'%d'", 부동 소수점 형식을 표현하기 위해서 \verb"'%g'",
문자열 형식을 표현하기 위해서 \verb"'%s'"을 사용한 사례다. 
여기서 왜 부동 소수점 형식이  \verb"'%f'"대신에 \verb"'%g'"인지는 질문하지 말아주세요.

\beforeverb
\begin{verbatim}
>>> 'In %d years I have spotted %g %s.' % (3, 0.1, 'camels')
'In 3 years I have spotted 0.1 camels.'
\end{verbatim}
\afterverb
%

튜플 요소 숫자는 문자열 서식 순서의 숫자와 일치해야 하고, 요소의 자료형(type)도 서식 순서와 일치해야 한다.

\index{예외 (exception)!자료형 오류 (TypeError)}
\index{자료형 오류 (TypeError)}

\beforeverb
\begin{verbatim}
>>> '%d %d %d' % (1, 2)
TypeError: not enough arguments for format string
>>> '%d' % 'dollars'
TypeError: illegal argument type for built-in operation
\end{verbatim}
\afterverb
%

상기 첫 예제는 충분한 요소 개수가 되지 않고, 두 번째 예제는 자료형이 맞지 않는다.

서식 연산자는 강력하지만, 사용하기가 어럽다.
더 많은 정보는 \url{docs.python.org/lib/typesseq-strings.html}에서 찾을 수 있다.

% You can specify the number of digits as part of the format sequence.
% For example, the sequence \verb"'%8.2f'"
% formats a floating-point number to be 8 characters long, with
% 2 digits after the decimal point:

% \beforeverb
% \begin{verbatim}
% >>> '%8.2f' % 3.14159
% '    3.14'
% \end{verbatim}
% \afterverb
% %
% The result takes up eight spaces with two
% digits after the decimal point.  


\section{    디버깅}
\index{디버깅 (debugging)}

프로그램을 작성하면서 배양해야 하는 기술은 항상 자신에게 질문을 하는 것이다.
''여기서 무엇이 잘못 될 수 있을까?'' 혹은 ''내가 작성한 완벽한 프로그램을 망가뜨리기 위해 사용자는 무슨 엄청난 일을 할 것인가?"

예를 들어 앞장의 반복 {\tt while} 루프를 시연하기 위해 사용한 프로그램을 살펴봅시다.

\beforeverb
\begin{verbatim}
while True:
    line = raw_input('> ')
    if line[0] == '#' :
        continue
    if line == 'done':
        break
    print line

print 'Done!'
\end{verbatim}
\afterverb
%

사용자가 입력값으로 빈 공백 줄을 입력하게 될 때 무엇이 발생하는지 살펴봅시다.

\beforeverb
\begin{verbatim}
> hello there
hello there
> # don't print this
> print this!
print this!
> 
Traceback (most recent call last):
  File "copytildone.py", line 3, in <module>
    if line[0] == '#' :
\end{verbatim}
\afterverb
%
빈 공백줄이 입력될 때까지 코드는 잘 작동합니다. 
그리고 나서, 0 번째 문자가 없어서 트레이스백(traceback)이 발생한다. 
입력줄이 비어있을 때, 코드 3번째 줄을 ''안전''하게 만드는 두 가지 방법이 있다.

하나는 빈 문자열이면 {\tt 거짓(False)}을 반환하도록 {\tt startswith} 메쏘드를 사용하는 것이다.

\beforeverb
\begin{verbatim}
    if line.startswith('#') :
\end{verbatim}
\afterverb
%
\index{가디언 패턴 (guardian pattern)}
\index{패턴 (pattern)!가디언 (guardian)}

{\bf 가디언 패턴(guardian pattern)}을 사용한 {\tt if}문으로 문자열에 적어도 하나의 문자가 있는 경우만 두번째 논리 표현식이 평가되도록 코드를 작성한다.

\beforeverb
\begin{verbatim}
    if len(line) > 0 and line[0] == '#' :
\end{verbatim}
\afterverb
%

\section{    용어정의}

\begin{description}

\item[계수기(counter):] 무언가를 계수하기 위해서 사용되는 변수로 일반적으로 0 으로 초기화하고 나서 증가한다.
\index{계수기 (counter)}

\item[빈 문자열(empty string):] 두 인용부호로 표현되고, 어떤 문자도 없고 길이가 0 인 문자열.
\index{빈 문자열 (empty string)}

\item[서식 연산자(format operator):] 서식 문자열과 튜플을 받아, 서식 문자열에 지정된 서식으로 튜플 요소를 포함하는 문자열을 생성하는 연산자, {\tt \%}.
\index{서식 연산자 (format operator)}
\index{연산자 (operator)!서식 (format)}

\item[서식 순서(format sequence):] {\tt \%d}처럼 어떤 값의 서식으로 표현되어야 하는지를 명세하는 ''서식 문자열'' 문자 순서. 
\index{서식 순서 (format sequence)}

\item[서식 문자열(format string):] 서식 순서를 포함하는 서식 연산자와 함께 사용되는 문자열.
\index{서식 문자열 (format string)}

\item[플래그(flag):] 조건이 참인지를 표기하기 위해 사용하는 불 변수(boolean variable)
\index{플래그 (flag)}

\item[호출(invocation):] 메쏘드를 호출하는 명령문.
\index{호출 (invocation)}

\item[불변(immutable):] 순서의 항목에 대입할 수 없는 특성.
\index{불변성 (immutability)}

\item[인덱스(index):] 문자열의 문자처럼 순서(sequence)에 항목을 선택하기 위해 사용되는 정수 값.
\index{인덱스 (index)}

\item[항목(item):] 순서에 있는 값의 하나.
\index{항목 (item)}

\item[메쏘드(method):] 객체와 연관되어 점 표기법을 사용하여 호출되는 함수.
\index{메쏘드 (method)}

\item[객체(object):] 변수가 참조하는 무엇. 지금은 ''객체''와 ''값''을 구별없이 사용한다.
\index{객체 (object)}

\item[검색(search):] 찾고자 하는 것을 찾았을 때 멈추는 운행법 패턴.
\index{검색 패턴 (search pattern)}
\index{패턴 (pattern)!검색 (search)}

\item[순서(sequence):] 정돈된 집합. 즉, 정수 인텍스로 각각의 값이 확인되는 값의 집합.
\index{순서 (sequence)}

\item[슬라이스(slice):] 인텍스 범위로 지정되는 문자열 부분.
\index{슬라이스 (slice)}

\item[운행법(traverse):] 순서(sequence)의 항목을 반복적으로 훑기, 각각에 대해서는 동일한 연산을 수행.
\index{운행법 (traversal)}

\end{description}


\section{    연습문제}

\begin{ex}
다음 문자열을 파이썬 코드를 작성하세요.

\beforeverb
\begin{alltt}
str = 'X-DSPAM-Confidence: {\bf 0.8475}'
\end{alltt}
\afterverb

{\tt find} 메쏘드와 문자열 슬라이싱을 사용하여 콜론(:) 문자 뒤 문자열을 뽑아내고 
{\tt float} 함수를 사용하여 뽑아낸 문자열을 부동 소수점 숫자로 변환하세요.

\end{ex}


\begin{ex}
\index{문자열 메쏘드 (string method)}
\index{메쏘드 (method)!문자열 (string)}

\url{https://docs.python.org/2.7/library/stdtypes.html#string-methods}에서 문자열 메쏘드 문서를 읽어보세요.
어떻게 동작하는가를 이해도를 확인하기 위해서 몇개를 골라 실험을 해보세요.
{\tt strip}과 {\tt replace}가 특히 유용합니다.

문서는 좀 혼동스러울 수 있는 구문을 사용합니다.
예를 들어, \verb"find(sub[, start[, end]])"의 꺾쇠기호는 선택(옵션) 인수를 나타냅니다.
그래서, {\tt sub}은 필수지만, {\tt start}은 선택 사항이고, 만약 {\tt start}가 인자로 포함된다면, {\tt end}는 선택이 된다.  
\end{ex}

