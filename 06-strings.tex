% LaTeX source for ``Python for Informatics: Exploring Information''
% Copyright (c)  2010-  Charles R. Severance, All Rights Reserved

\chapter{문자열}
\label{strings}


\section{문자열은 순열이다.}
\index{sequence}
\index{character}
\index{bracket operator}
\index{operator!bracket}

문자열은 문자의 순열이다. 대괄호 연산자로 한번에 하나씩 문자에 접근할 수 있다.

\beforeverb
\begin{verbatim}
>>> fruit = 'banana'
>>> letter = fruit[1]
\end{verbatim}
\afterverb
%
\index{index}

두 번째 명령문은 변수 {\tt fruit}에서 1번 위치의 문자를 추출하여 변수 {\tt letter}에 할당한다.
대괄호의 표현식을 {\bf 인덱스(index)}라고 부른다. 인덱스는 순열의 무슨 문자를 사용자가 원하는지 표시한다.

하지만, 여려분이 기대한 것을 얻지 못합니다.

\beforeverb
\begin{verbatim}
>>> print letter
a
\end{verbatim}
\afterverb
%

대부분의 사람에게 \verb"'banana'"의 첫 분자는 {\tt a}가 아니라 {\tt b}입니다.
하지만, 파이썬에서 인텍스는 문자열의 처음부터 값이 시작됩니다. 첫 글자의 시작 값(오프셋, offset)은 0입니다.

\beforeverb
\begin{verbatim}
>>> letter = fruit[0]
>>> print letter
b
\end{verbatim}
\afterverb
%

{\tt b}가 \verb"'banana'"의 0번째 문자가 되고 {\tt a}가 첫번째, {\tt n}이 두번째 문자가 됩니다.

\beforefig
\centerline{\includegraphics[height=0.50in]{figs2/string.eps}}
\afterfig

\index{index!starting at zero}
\index{zero, index starting at}

인덱스로 문자와 연산자를 포함하는 어떤 표현식도 사용가능지만, 인덱스의 값은 정수여야 합니다.
정수가 아닌 경우 다음과 같은 결과를 얻게 됩니다.

\index{index}
\index{exception!TypeError}
\index{TypeError}

\beforeverb
\begin{verbatim}
>>> letter = fruit[1.5]
TypeError: string indices must be integers
\end{verbatim}
\afterverb
%

\section{{\tt len}함수를 사용하여 문자열의 길이 구하기}

\index{len function}
\index{function!len}

{\tt len}은 문자열의 문자 갯수를 반환하는 내장함수다.

\beforeverb
\begin{verbatim}
>>> fruit = 'banana'
>>> len(fruit)
6
\end{verbatim}
\afterverb
%

문자열의 가장 마지막 문자를 얻기 위해서, 아래와 같이 시도하려고 할 것입니다.

\index{exception!IndexError}
\index{IndexError}

\beforeverb
\begin{verbatim}
>>> length = len(fruit)
>>> last = fruit[length]
IndexError: string index out of range
\end{verbatim}
\afterverb
%

{\tt IndexError}의 이유는 {\tt 'banana'}에 6번 인텍스의 문자가 없기 때문입니다.
0에서부터 시작했기 때문에 6개의 문자는 0~5로 번호가 매겨졌습니다. 마지막 문자를 얻기 위해서 
{\tt length}에서 1을 빼야 합니다.


\beforeverb
\begin{verbatim}
>>> last = fruit[length-1]
>>> print last
a
\end{verbatim}
\afterverb
%

대안으로 문자의 끝에서 역으로 수를 세는 음의 인텍스를 사용할 수 있습니다. 

{\tt fruit[-1]}는 마지막 문자를 {\tt fruit[-2]}는 끝에서 두 번째 등등 활용할 수 있습니다.

\index{index!negative}
\index{negative index}


\section{루프를 사용한 문자열 운행법}
\label{for}

\index{traversal}
\index{loop!traversal}
\index{for loop}
\index{loop!for}
\index{statement!for}
\index{traversal}

많은 연산의 경우 문자열을 한번에 한 문자씩 처리하는 합니다. 종종 처음에서 시작해서, 차례로 각 문자를 선택하고,
선택된 문자에 임의의 연산을 수행하고, 끝까지 계속합니다. 이런 처리 패턴을 {\bf 운행법(traversal)}라고 합니다.
운행법을 작성하는 한 방법은 {\tt while} 루프입니다.

\beforeverb
\begin{verbatim}
index = 0
while index < len(fruit):
    letter = fruit[index]
    print letter
    index = index + 1
\end{verbatim}
\afterverb
%
{\tt while} 루프가 문자열을 운행하여 문자열을 한줄에 한자씩 화면에 출력합니다.
루프 조건이 {\tt index < len(fruit)}이여서, {\tt index}가 문자열의 길와 같을 때,
조건은 거짓이 되고, 루프의 몸통 부문은 실행이 되지 않습니다. 
파이썬이 접근한 마지막 문자는 문자열의 마지막 문자인 {\tt len(fruit)-1} 인텍스의 문자입니다.

\begin{ex}
문자열의 마지막 문자에서 시작해서,문자열의 처음으로 역진행하면서 한줄에 한자씩 화면에 출력하는 {\tt while} 루프를 작성하세요.
\end{ex}

운행법을 작성하는 다른 방법은 {\tt for} 루프입니다.

\beforeverb
\begin{verbatim}
for char in fruit:
    print char
\end{verbatim}
\afterverb
%

루프를 매번 반복할 때, 문자열의 다음 문자가 변수 {\tt char}에 할당됩니다. 루프는 더 이상 남겨진 문자가 없을 때까지 계속 실행됩니다.

\section{문자열 조각}
\label{slice}

\index{slice operator}
\index{operator!slice}
\index{index!slice}
\index{string!slice}
\index{slice!string}

문자열의 일부분을 {\bf 문자열 조각(slice)}이라고 합니다. 문자열 조각을 선택하는 것은 문자를 선택하는 것과 유사합니다.

\beforeverb
\begin{verbatim}
>>> s = 'Monty Python'
>>> print s[0:5]
Monty
>>> print s[6:13]
Python
\end{verbatim}
\afterverb
%
{\tt [n:m]} 연산자는 n번째 문자부터 m번째 문자까지의 문자열 - 첫 번째는 포함하지만 마지막은 제외 - 부분을 반환합니다.

콜론 앞의 첫 인텍스를 생략하면, 문자열 조각은 문자열의 처음부터 시작합니다.
두 번째 인텍스를 생략하면, 문자열 조각은 문자열의 끝까지 갑니다. 

\beforeverb
\begin{verbatim}
>>> fruit = 'banana'
>>> fruit[:3]
'ban'
>>> fruit[3:]
'ana'
\end{verbatim}
\afterverb
%

만약 첫번째 인텍스가 두번째보다 크거나 같은 경우 결과는 두 인용부호로 표현되는 {\bf 빈 문자열(empty string)}이 됩니다.

\index{quotation mark}

\beforeverb
\begin{verbatim}
>>> fruit = 'banana'
>>> fruit[3:3]
''
\end{verbatim}
\afterverb
%

빈 문자열은 어떠한 문자도 포함하고 있지 않아서 길이가 0 이지만, 이외에 다른 문자열과 동일합니다.

\begin{ex}
{\tt fruit}이 문자열로 주어졌을 때, {\tt fruit[:]}의 의미는 무엇인가요?

\index{copy!slice}
\index{slice!copy}


\end{ex}


\section{문자열은 불변이다.}
\index{mutability}
\index{immutability}
\index{string!immutable}

문자열의 문자를 변경하려는 의도로 할당문의 왼쪽편에 {\tt []} 연산자를 사용하고 싶은 유혹이 있을 것입니다.
예를 들어 다음과 같습니다.

\index{TypeError}
\index{exception!TypeError}

\beforeverb
\begin{verbatim}
>>> greeting = 'Hello, world!'
>>> greeting[0] = 'J'
TypeError: object does not support item assignment
\end{verbatim}
\afterverb
%

이 경우 ''개체''는 문자열이고, 할당하고자 하는 문자는 ''항목''이다. 지금 당장은
{\bf 개체}는 값과 같은 것이지만, 후에 좀더 정의를 상세화할 것입니다.
{\bf 항목}은 순열의 값중의 하나입니다.

\index{object}
\index{item assignment}
\index{assignment!item}
\index{immutability}

오류의 이유는 문자열이 {\bf 불변(immutable)}이기 때문입니다. 따라서 존재하는 문자열을 바꿀 수 없다는 의미입니다.
최선은 원래 문자열에 변화를 준 새로운 문자열을 생성하는 것입니다.

\beforeverb
\begin{verbatim}
>>> greeting = 'Hello, world!'
>>> new_greeting = 'J' + greeting[1:]
>>> print new_greeting
Jello, world!
\end{verbatim}
\afterverb
%
새로운 첫 문자에 {\tt greeting} 문자열 조각을 연결해서, 원래 문자열에는 어떤한 영향도 주지 않는 새로운 문자열을 만들었습니다.

\index{concatenation}

\section{루프 돌기(looping)와 세기(counting)}
\label{counter}

\index{counter}
\index{counting and looping}
\index{looping and counting}
\index{looping!with strings}

다음 프로그램은 문자열에 문자 {\tt a}가 나타나는 횟수를 셉니다.

\beforeverb
\begin{verbatim}
word = 'banana'
count = 0
for letter in word:
    if letter == 'a':
        count = count + 1
print count
\end{verbatim}
\afterverb
%

상기 프로그램은 {\bf 카운터(counter)}라고 부르는 또다른 연산 패턴을 보여줍니다. 
변수 {\tt count}는 0으로 초기화 되고, 매번 {\tt a}를 찾을 때마나 증가합니다.
루프를 빠져나갔을 때, {\tt count}는 결과 값, {\tt a}가 나타난 총 횟수를 담고 있습니다.

\begin{ex}
\index{encapsulation}

상기 코드를 캡슐화(encapsulation)하여 문자열과 문자를 인수로 받는 {\tt count}라는 함수를 작성해서 일반화하세요.

\end{ex}

\section{{\tt in} 연산자}
\label{inboth}

\index{in operator}
\index{operator!in}
\index{boolean operator}
\index{operator!boolean}

연산자 {\tt in}은 불 연산자로 두개의 문자열을 받아, 첫 번째 문자열이 두 번째 문자열의 일부이면 {\tt 참(True)}을 반환한다.

\beforeverb
\begin{verbatim}
>>> 'a' in 'banana'
True
>>> 'seed' in 'banana'
False
\end{verbatim}
\afterverb
%

\section{문자열 비교}

\index{string!comparison}
\index{comparison!string}

비교 연산자도 문자열에서 동작합니다. 두 문자열이 같은지를 살펴봅시다.

\beforeverb
\begin{verbatim}
if word == 'banana':
    print  'All right, bananas.'
\end{verbatim}
\afterverb
%

다른 비교 연산자는 단어를 알파벳 순으로 정렬하는데 유용하다.

\beforeverb
\begin{verbatim}
if word < 'banana':
    print 'Your word,' + word + ', comes before banana.'
elif word > 'banana':
    print 'Your word,' + word + ', comes after banana.'
else:
    print 'All right, bananas.'
\end{verbatim}
\afterverb
%

파이썬은 사람과 동일하는 방식으로 대문자와 소문자를 다루지 않습니다.
모든 대문자는 소문자 앞에 위치합니다.

\beforeverb
\begin{verbatim}
Your word, Pineapple, comes before banana.
\end{verbatim}
\afterverb
%

이러한 문제를 다루는 일반적인 방식은 비교연산을 수행하기 전에 문자열을 표준 포맷으로 예를 들어 모두 소문자, 변환하는 것입니다.
''Pineapple''로 무장한 사람들로부터 여러분을 보호하는 경우를 명심하세요.


\section{{\tt string} 메쏘드}
문자열은 파이썬 {\bf 개체(objects)}의 한 예이다. 개체는 데이터(실제 문자열 자체)와 {\bf 메쏘드(methods)}를 담고 있다.
메쏘드는 개체내부에 내장되고 개체의 어떤 {\bf 인스턴스(instance)}에도 사용될 수 있는 효과적인 함수다.

파이썬은 개체에 대해서 이용가능한 메쏘드를 보여주는 {\tt dir} 함수가 있다.
{\tt type} 함수는 개체의 형(type)을 보여주고, {\tt dir}은 개체의 사용될 수 있는 메쏘드를 보여준다.

\beforeverb
\begin{verbatim}
>>> stuff = 'Hello world'
>>> type(stuff)
<type 'str'>
>>> dir(stuff)
['capitalize', 'center', 'count', 'decode', 'encode', 
'endswith', 'expandtabs', 'find', 'format', 'index', 
'isalnum', 'isalpha', 'isdigit', 'islower', 'isspace', 
'istitle', 'isupper', 'join', 'ljust', 'lower', 'lstrip', 
'partition', 'replace', 'rfind', 'rindex', 'rjust', 
'rpartition', 'rsplit', 'rstrip', 'split', 'splitlines', 
'startswith', 'strip', 'swapcase', 'title', 'translate', 
'upper', 'zfill']
>>> help(str.capitalize)
Help on method_descriptor:

capitalize(...)
    S.capitalize() -> string
    
    Return a copy of the string S with only its first character
    capitalized.
>>>
\end{verbatim}
\afterverb
%

{\tt dir} 함수가 메쏘드를 보여주고, 메쏘드에 대한 간단한 문서를 {\tt help}를 사용할 수 있지만,
문자열 메쏘드에 대한 좀더 좋은 문서 정보는 \url{docs.python.org/library/string.html}에서 찾을 수 있다.

인수를 받고 값을 반환한다는 점에서 {\bf 메쏘드(method)}를 호출하는 것은 함수를 호출하는 것과 유사하지만, 구문은 다르다.
구분자로 점을 사용해서 변수명에 메쏘드명을 붙여 메쏘드를 호출한다.

예를 들어, {\tt upper} 메쏘드는 문자열을 받아 모두 대문자로 변경된 새로운 문자열을 반환한다.

\index{method}
\index{string!method}

함수 구문 {\tt upper(word)} 대신에, {\tt word.upper()} 메쏘드 구문을 사용한다.

\index{dot notation}

\beforeverb
\begin{verbatim}
>>> word = 'banana'
>>> new_word = word.upper()
>>> print new_word
BANANA
\end{verbatim}
\afterverb
%

이런 형태의 점 표기법은 메쏘드 이름({\tt upper})과 메쏘드가 적용되는 문자열 이름({\tt word})을 명세한다.
빈 괄호는 메쏘드가 인수를 갖지 않는 것을 나타낸다.

\index{parentheses!empty}

메쏘드를 부르는 것을 {\bf 호출(invocation)}이라고 부른다. 상기의 경우, {\tt word}에 {\tt upper} 메쏘드를 호출한다고 말한다.

\index{invocation}

예를 들어, 문자열안에 한 문자의 위치를 찾는 {\tt find}라는 문자열 메쏘드가 있다.

\beforeverb
\begin{verbatim}
>>> word = 'banana'
>>> index = word.find('a')
>>> print index
1
\end{verbatim}
\afterverb
%

상기 예제에서, {\tt word} 문자열의 {\tt find} 메쏘드를 호출하여 매개 변수로 찾고 있는 문자를 넘긴다.

{\tt find} 메쏘드는 문자뿐만 아니라 부분 문자열도 찾을 수 있다.

\beforeverb
\begin{verbatim}
>>> word.find('na')
2
\end{verbatim}
\afterverb
%

검색 시작 위치를 지정하는 두 번째 인텍스를 인수로 갖을 수도 있다.

\index{optional argument}
\index{argument!optional}

\beforeverb
\begin{verbatim}
>>> word.find('na', 3)
4
\end{verbatim}
\afterverb
%

한가지 자주 있는 적업은 {\tt strip} 메쏘드를 사용해서 문자열의 시작과 끝의 공백(공백 여러개, 탭, 새줄)을 제거하는 것이다.

\beforeverb
\begin{verbatim}
>>> line = '  Here we go  '
>>> line.strip()
'Here we go'
\end{verbatim}
\afterverb
%

{\bf startswith} 메쏘드는 참, 거짓 같은 불 값을 반환한다.

\beforeverb
\begin{verbatim}
>>> line = 'Please have a nice day'
>>> line.startswith('Please')
True
>>> line.startswith('p')
False
\end{verbatim}
\afterverb
%

{\bf startswith}가 대소문자를 구별하는 것을 요구하기 때문에 {\tt lower} 메쏘드를 사용해서 임의의 검증을 수행하기 전에, 
한 줄을 입력받아 모두 소문자로 변환하는 것이 필요하다.

\beforeverb
\begin{verbatim}
>>> line = 'Please have a nice day'
>>> line.startswith('p')
False
>>> line.lower()
'please have a nice day'
>>> line.lower().startswith('p')
True
\end{verbatim}
\afterverb
%

마지막 예제에서 결과 문자열이 문자 ''p''로 시작하는지를 검증하기 위해서, 
{\tt lower} 메쏘드가 호출되고 나서 바로 {\tt startswith} 메쏘드를 사용한다.
순서만 주의깊게 다룬다면, 한줄에 여러개의 메쏘드를 호출할 수 있다.

\begin{ex}
\index{count method}
\index{method!count}

앞선 예제와 유사한 함수인 {\tt count}로 불리는 문자열 메쏘드가 있다.
\url{docs.python.org/library/string.html}에서 {\tt count} 메쏘드에 대한 문서를 읽고,
문자열 \verb"'banana'"의 문자가 몇 개인지 세는 메쏘드 호출 프로그램을 작성하세요.

\end{ex}

\section{문자열 파싱(Parsing)}

종종, 문자열을 들여다 보고 부속 문자열(substring)을 찾고 싶다. 예를 들어,
아래와 같은 형식으로 구성된 일련의 자료 라인이 주어졌다고 가정하면,

\beforeverb
\begin{alltt}
From stephen.marquard@{\bf uct.ac.za} Sat Jan  5 09:14:16 2008
\end{alltt}
\afterverb

각 라인의 뒤쪽 전자우편 주소({\tt uct.ac.za})만 뽑아내고 싶을 것이다.
{\tt find} 메쏘드와 문자열 조각(string sliceing)을 사용해서 작업을 수행할 수 있다.

우선, 문자열에서 골뱅이(@, at-sign) 기호의 위치를 찾는다. 그리고 
골뱅이 기호 \emph{뒤} 첫 공백 위치를 찾는다. 그리고 나서 찾고자 하는 부속 문자열을 뽑아내기 위해서 문자열 조각을 사용한다.

\beforeverb
\begin{verbatim}
>>> data = 'From stephen.marquard@uct.ac.za Sat Jan  5 09:14:16 2008'
>>> atpos = data.find('@')
>>> print atpos
21
>>> sppos = data.find(' ',atpos)
>>> print sppos
31
>>> host = data[atpos+1:sppos]
>>> print host
uct.ac.za
>>> 
\end{verbatim}
\afterverb
%

{\tt find} 메쏘드를 사용해서 찾고자 하는 문자열의 시작 위치를 명세한다. 
문자열 조각낼(slicing) 때, 골뱅기 기호 뒤부터 빈 공백을 \emph{포함하지 않는} 위치까지 문자열을 뽑아낸다.  

{\tt find} 메쏘드에 대한 문서는 \url{docs.python.org/library/string.html}에서 참조 가능하다.

\section{형식 연산자}

\index{format operator}
\index{operator!format}

{\bf 형식 연산자(format operator)}, {\tt \%}는 문자열의 일부를 변수에 저장된 값으로 바꿔 문자열을 구성한다.
정수에 포맷 연산자가 적용될 때, {\tt \%}는 나머지 연산가 된다. 하지만 첫 피연산자가 문자열이면,  {\tt \%}은 포맷 연산자가 된다.

\index{format string}

첫 피연산자는 {\bf 형식 문자열 format string}로 두번째 피 연산자가 어떤 형식으로 표현되는지를 명세하는 하나 혹은 그 이상의 
{\bf 형식 열 format sequence}을 담고 있다. 결과값은 문자열이다.

\index{format sequence}

예를 들어, 형식 열 \verb"'%d'"의 의미는 두번째 피연산자가 정수 형식으로 표현됨을 뜻한다. ({\tt d}는 ``decimal''를 나타낸다.)

\beforeverb
\begin{verbatim}
>>> camels = 42
>>> '%d' % camels
'42'
\end{verbatim}
\afterverb
%

결과는 \verb"'42'" 문자열로 정수 {\tt 42}와 혼동하면 안 된다.

형식 열은 문자열 어디에서도 나타날 수 있어서 문장 중간에 값을 임베드(embed)할 수 있다.

\beforeverb
\begin{verbatim}
>>> camels = 42
>>> 'I have spotted %d camels.' % camels
'I have spotted 42 camels.'
\end{verbatim}
\afterverb
%

만약 문자열의 형식 열이 하나 이상이라면, 두번째 인수는 튜플(tuple)이 된다.
형식 열 각각은 튜플의 요소와 순서대로 일치된다.

다음 예제는 정수 형식을 표현하기 위해서 \verb"'%d'", 부동 소수점 형식을 표현하기 위해서 \verb"'%g'",
문자열 형식을 표현하기 위해서 \verb"'%s'"을 사용한 것을 보여준다. 여기서 왜 부동 소수점 형식이  \verb"'%f'"대신에 \verb"'%g'"인지는 질문하지 말아주세요.

\beforeverb
\begin{verbatim}
>>> 'In %d years I have spotted %g %s.' % (3, 0.1, 'camels')
'In 3 years I have spotted 0.1 camels.'
\end{verbatim}
\afterverb
%

튜플 요소 숫자는 문자열의 형식 열의 숫자와 일치해야 하고, 요소의 형도 형식 열과 일치해야 한다.

\index{exception!TypeError}
\index{TypeError}

\beforeverb
\begin{verbatim}
>>> '%d %d %d' % (1, 2)
TypeError: not enough arguments for format string
>>> '%d' % 'dollars'
TypeError: illegal argument type for built-in operation
\end{verbatim}
\afterverb
%

첫 예제는 충분한 요소 개수가 않돼고, 두 번째 예제는 형식이 맞지 않는다.

형식 연산자는 강력하지만, 사용하기가 여럽다. 더 많은 정보는 
\url{docs.python.org/lib/typesseq-strings.html}에서 찾을 수 있다.

% You can specify the number of digits as part of the format sequence.
% For example, the sequence \verb"'%8.2f'"
% formats a floating-point number to be 8 characters long, with
% 2 digits after the decimal point:

% \beforeverb
% \begin{verbatim}
% >>> '%8.2f' % 3.14159
% '    3.14'
% \end{verbatim}
% \afterverb
% %
% The result takes up eight spaces with two
% digits after the decimal point.  


\section{디버깅}
\index{debugging}

프로그램을 작성하면서 배양해야 하는 기술은 항상 자신에게 질문을 하는 것이다.
''여기서 무엇이 잘못 될 수 있을까?'' 혹은 ''사용자가 내가 작성한 완벽한 프로그램을 망가뜨리기 위해 무슨 엄청난 일을 할 것인가?"

예를 들어 앞장의 반복 {\tt while} 루프를 시연하기 위해 사용한 프로그램을 살펴봅시다.

\beforeverb
\begin{verbatim}
while True:
    line = raw_input('> ')
    if line[0] == '#' :
        continue
    if line == 'done':
        break
    print line

print 'Done!'
\end{verbatim}
\afterverb
%

사용자가 입력값으로 빈 공백 줄을 입력하게 될때 무엇이 발생하는지 살펴봅시다.

\beforeverb
\begin{verbatim}
> hello there
hello there
> # don't print this
> print this!
print this!
> 
Traceback (most recent call last):
  File "copytildone.py", line 3, in <module>
    if line[0] == '#' :
\end{verbatim}
\afterverb
%
빈 공백줄이 입력될 때까지 코드는 잘 작동합니다. 
0번째 문자가 없어서 트레이스백(traceback)이 발생한다. 입력줄이 비여있을 때, 코드 3번째 줄을 ''안전''하게 만드는 두가지 방법이 있다.

{\tt startswith} 메쏘드를 사용해서 빈 문자열이면  {\tt 거짓(False)}을 반환한다.

\beforeverb
\begin{verbatim}
    if line.startswith('#') :
\end{verbatim}
\afterverb
%
\index{guardian pattern}
\index{pattern!guardian}

{\bf 가디언 패턴(guardian pattern)}을 사용한 {\tt if}문으로 문자열에 적어도 하나의 문자가 있는 경우만 두번째 논리 표현식이 평가되도록 하는 코드를 작성한다.

\beforeverb
\begin{verbatim}
    if len(line) > 0 and line[0] == '#' :
\end{verbatim}
\afterverb
%

\section{용어정의}

\begin{description}

\item[카운터(counter):] 무언가를 세기 위해서 사용되는 변수로 일반적으로 0으로 초기화 되고 증가한다.
\index{counter}

\item[빈 문자열(empty string):] 두 인용부호로 표현되는 어떤 문자도 없고 길이가 0인 문자열.
\index{empty string}

\item[형식 연산자(format operator):] 
형식 문자열과 튜플을 받아, 형식 문자열에 지정된 형식으로 튜플 요소를 포함하는 문자열을 생성하는 연산자, {\tt \%}.
\index{format operator}
\index{operator!format}

\item[행식 열(format sequence):] 
{\tt \%d}처럼 어떻게 값이 형식적으로 표현되어야 하는지를 명세하는 ''형식 문자열''의 문자의 열. 
\index{format sequence}

\item[형식 문자열(format string):] 형식 열을 포함하는 형식 연산자와 함께 사용되는 문자열.
\index{format string}

\item[플래그(flag):] 조건이 참인지를 표기위해 사용하는 불 변수(boolean variable)
\index{flag}

\item[호출(invocation):] 메쏘드를 호출하는 명령문.
\index{invocation}

\item[불변(immutable):] 열의 항목에 할당할 수 없는 특성.
\index{immutability}

\item[인덱스(index):] 문자열의 문자처럼 열의 항목을 선택하기 위해 사용되는 정수 값.
\index{index}

\item[항목(item):] 열에 있는 값의 하나.
\index{item}

\item[메쏘드(method):] 개체와 연관되어 점 표기법을 사용하여 호출되는 함수.
\index{method}

\item[개체(object):] 변수가 참조하는 무엇. 지금에서는 ''개체''와 ''값''을 구별없이 사용한다.
\index{object}

\item[검색(search):] 찾고자 하는 것을 찾았을 때 멈추는 운행법 패턴.
\index{search pattern}
\index{pattern!search}

\item[열(sequence):] 정돈된 세트. 즉, 정수 인텍드로 각각의 값이 확인되는 값의 집합.
\index{sequence}

\item[조각(slice):] 인텍스의 범위로 지정된 문자열 부분.
\index{slice}

\item[운행법(traverse):] 열의 항목을 반복적으로 훑기, 각각에 대해서는 동일한 연산을 수행.
\index{traversal}

\end{description}


\section{연습문제}

\begin{ex}
아래 문자열을 파이썬 코드를 작성하세요.

\beforeverb
\begin{alltt}
str = 'X-DSPAM-Confidence: {\bf 0.8475}'
\end{alltt}
\afterverb

{\tt find} 메쏘드와 문자열 조각내기를 사용하여 콜론(:) 문자 뒤의 문자열을 뽑아내고 
{\tt float} 함수를 사용하여 뽑아낸 문자열을 부동 소수점 숫자로 변환하세요.

\end{ex}


\begin{ex}
\index{string method}
\index{method!string}

\url{https://docs.python.org/2.7/library/stdtypes.html#string-methods}에서 문자열 메쏘드 문서를 읽어보세요.
어떻게 동작하는가를 이해도를 확인하기 위해서 몇개를 골라 실험을 해보세요.
{\tt strip}과 {\tt replace}가 특히 유용합니다.

문서는 
문서는 좀 혼동스러울 수 있는 구문을 사용합니다.
예를 들어, \verb"find(sub[, start[, end]])"의 꺾쇠기호는 선택(옵션) 인수를 나타냅니다.
그래서, {\tt sub}은 필수지만, {\tt start}은 선택 사항이고, 만약 {\tt start}가 인수로 포함된다면, {\tt end}는 선택이 된다.  
\end{ex}

