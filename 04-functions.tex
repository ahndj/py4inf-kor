% LaTeX source for ``Python for Informatics: Exploring Information''
% Copyright (c)  2010-  Charles R. Severance, All Rights Reserved

\chapter{함수}
\label{funcchap}

\section{함수 호출}
\label{functionchap}
\index{함수 호출}
%\index{function call}

프로그래밍 문맥에서, {\bf 함수(function)}는 연산을 수행하는 명명된 일련의 문장이다. 
함수를 정의할 때, 이름과 일련의 문장을 명기한다. 
나중에, 함수를 이름으로 ''호출(call)''한다.
이미 {\bf 함수 호출(function call)}의 예제를 살펴봤다.

\beforeverb
\begin{verbatim}
>>> type(32)
<type 'int'>
\end{verbatim}
\afterverb
%

함수명은 {\tt type}이다. 
괄호안의 표현식을 함수의 {\bf 인자(argument)}라고 한다. 
인자는 함수 입력으로 함수 내부로 전달되는 값이나 변수이다. 
앞선 {\tt type} 함수에 대한 결과는 인자의 형(type)이다.

\index{괄호!인자 입력}
%\index{parentheses!argument in}

통상 함수가 인자를 ''받아'' 결과를 ''반환''한다. 결과를 {\bf 결과값(return value)}이라 부른다.

\index{인자}
\index{반환값}
%\index{argument}
%\index{return value}

\section{내장(Built-in) 함수}

함수를 정의할 필요없이 사용할 수 있는 내장함수가 파이썬에는 많다.
공통의 문제를 해결할 수 있는 함수를 파이썬을 창시자(Guido van Rossum)가 작성해서 누구나 사용할 수 있도록 파이썬에 포함했다.

{\tt max}와 {\tt min} 함수는 리스트 최소값과 최대값을 각각 계산해서 출력한다.

\beforeverb
\begin{verbatim}
>>> max('Hello world')
'w'
>>> min('Hello world')
' '
>>>
\end{verbatim}
\afterverb
%

{\tt max} 함수는 문자열의 "가장 큰 문자", 상기 예제에서 ''w'', 
{\tt min} 함수는 최소 문자를, 상기 예제에서는 공백, 출력한다.

매우 자주 사용되는 또 다른 내장 함수는 얼마나 많은 항목이 있는지 출력하는 {\tt len} 함수가 있다.
만약 {\tt len} 함수의 인수가 문자열이면 문자열에 있는 문자 갯수를 반환한다.

\beforeverb
\begin{verbatim}
>>> len('Hello world')
11
>>>
\end{verbatim}
\afterverb
%

이들 함수는 문자열에만 국한된 것이 아니라, 뒷장에서 보듯이 다양한 자료형에 사용된다.

내장함수 이름은 사전에 점유된 예약어로 취급해야 한다. 예를 들어 ''max''를 변수명으로 사용하지 말아야 한다.


\section{형(type) 변환 함수}
\index{변환!형(type)}
\index{형 변환}
%\index{conversion!type}
%\index{type conversion}

% from Elkner:
% comment on whether these things are _really_ functions?
% use max as an example of a built-in?

% my reply:
% they are on the list of ``built-in functions'' so I am
% willing to call them functions.

이런 형(type)에서 저런 형(type)으로 값을 변환하는 내장 함수가 파이썬에는 있다.
{\tt int}함수는 임의의 값을 입력받아 변환이 가능하면 정수형으로 변환하고, 그렇지 않으면 오류가 발생한다.

\index{int 함수}
\index{함수!int}
%\index{int function}
%\index{function!int}

\beforeverb
\begin{verbatim}
>>> int('32')
32
>>> int('Hello')
ValueError: invalid literal for int(): Hello
\end{verbatim}
\afterverb
%

{\tt int}는 부동 소수점 값을 정수로 변환할 수 있지만 소수점 이하를 절사한다.

\beforeverb
\begin{verbatim}
>>> int(3.99999)
3
>>> int(-2.3)
-2
\end{verbatim}
\afterverb
%

{\tt float}는 정수와 문자열을 부동 소수점으로 변환한다.

\index{float 함수}
\index{함수!float}
%\index{float function}
%\index{function!float}

\beforeverb
\begin{verbatim}
>>> float(32)
32.0
>>> float('3.14159')
3.14159
\end{verbatim}
\afterverb
%

마지막으로, {\tt str}은 인자를 문자열로 변환한다.

\index{str 함수}
\index{함수!str}
%\index{str function}
%\index{function!str}

\beforeverb
\begin{verbatim}
>>> str(32)
'32'
>>> str(3.14159)
'3.14159'
\end{verbatim}
\afterverb
%

\section{난수(Random numbers)}

\index{난수}
\index{숫자, 랜덤}
\index{결정적}
\index{의사 난수}
%\index{random number}
%\index{number, random}
%\index{deterministic}
%\index{pseudorandom}

동일한 입력을 받을 때, 대부분의 컴퓨터는 매번 동일한 출력값을 생성하기 때문에 {\bf 결정적(deterministic)}이라고 한다.
결정론이 대체로 좋은 것이다. 왜냐하면, 동일한 결과를 얻는데 동일한 계산을 기대하기 때문입니다. 
하지만, 어떤 응용프로그램에 대해서 컴퓨터가 예측불가능하길 바란다. 
게임이 좋은 예가 되고, 더 많은 예는 얼마든지 많다.

진실되게 프로그램을 비결정론적으로 만드는 것이 쉽지 않은 것으로 밝혀졌지만, 
적어도 비결정론적인 것처럼 보이게 하는 방법은 있다. 
{\bf 의사 난수(pseudorandom numbers)}를 생성하는 {\bf 알고리즘}을 사용하는 것이 방법 중의 하나다. 
의사 난수는 이미 결정된 연산에 의해서 생성된다는 점에서 진정한 의미의 난수는 아니지만, 
이렇게 생성된 숫자만 봐서는 진정한 난수와 구별하는 것은 불가능에 가깝다.

\index{random 모듈}
\index{모듈!random}
%\index{random module}
%\index{module!random}

{\tt random} 모듈안에 의사 난수를 생성하는 함수가 있다. 
(이하 의사 난수 대신 ''랜덤(random)''으로 간략히 부르기로 한다.)

\index{random 함수}
\index{함수!random}

{\tt random} 함수는 0.0 과 1.0 사이 부동 소수점 난수를 반환한다. {\tt random} 함수는 0.0 은 포함하지만 1.0은 포함하지 않는다.
매번 {\tt random} 함수를 호출할 때 마다, 이미 생성된 아주 긴 난수열에서 하나씩 하나씩 뽑아 쓰다. 
사례로 다음 반복문을 실행하자.

\beforeverb
\begin{verbatim}
import random

for i in range(10):
    x = random.random()
    print x
\end{verbatim}
\afterverb
%

상기 프로그램은 0.0 에서 1.0 구간(단,  1.0은 포함하지 않음)에서 10개 난수 리스트를 생성한다.

\beforeverb
\begin{verbatim}
0.301927091705
0.513787075867
0.319470430881
0.285145917252
0.839069045123
0.322027080731
0.550722110248
0.366591677812
0.396981483964
0.838116437404
\end{verbatim}
\afterverb
%

\begin{ex}
여러분의 컴퓨터에 프로그램을 실행해서, 어떤 난수가 생성되는지 살펴보세요.
한번 이상 프로그램을 실행하여 보고, 어떤 난수가 생성되는지 다시 살펴보세요.
\end{ex}

{\tt random} 함수는 난수를 다루는 많은 함수 중의 하나다.
{\tt randint} 함수는 {\tt 최저(low)}와 {\tt 최고(high)} 매개 변수를 입력받아 
{\tt 최저값(low)}과 {\tt 최고값(high)} 사이 (최저값과, 최저값 모두 포함) 정수를 반환한다.

\index{randint 함수}
\index{함수!randint}
%\index{randint function}
%\index{function!randint}

\beforeverb
\begin{verbatim}
>>> random.randint(5, 10)
5
>>> random.randint(5, 10)
9
\end{verbatim}
\afterverb
%

무작위로 특정 열에서 하나의 요소를 뽑아내기 위해, {\tt choice}를 사용한다.

\index{choice 함수}
\index{함수!choice}
%\index{choice function}
%\index{function!choice}

\beforeverb
\begin{verbatim}
>>> t = [1, 2, 3]
>>> random.choice(t)
2
>>> random.choice(t)
3
\end{verbatim}
\afterverb
%

또한 {\tt random} 모듈은 정규분포, 지수분포, 감마분포 및 기타 연속형 분포에서 난수를 생성하는 함수도 제공한다.

\section{수학 함수}
\index{수학 함수}
\index{함수, 수학}
\index{모듈}
\index{모듈 객체}
%\index{math function}
%\index{function, math}
%\index{module}
%\index{module object}

파이썬은 가장 친숙한 수학 함수를 제공하는 수학 모듈이 있다.
수학 모듈을 사용하기 전에, 수학 모듈 가져오기를 실행한다.

\beforeverb
\begin{verbatim}
>>> import math
\end{verbatim}
\afterverb
%

상기 명령문은 math 라는 {\bf 모듈 객체}를 생성한다. 
모듈 객체를 출력하면, 모듈 객체에 대한 정보를 얻을 수 있다.

\beforeverb
\begin{verbatim}
>>> print math
<module 'math' from '/usr/lib/python2.5/lib-dynload/math.so'>
\end{verbatim}
\afterverb
%

모듈 객체는 모듈에 정의된 함수와 변수를 담고 있다. 
함수 중에서 한 함수에 접근하기 위해서는, 모듈 이름과 함수 이름을 명시해야 하는데, 둘은 점(구두점)으로 구분된다.
이런 형식을 {\bf 점 표기법(dot notation)}이라고 부른다.

\index{점 표기법}
%\index{dot notation}

\beforeverb
\begin{verbatim}
>>> ratio = signal_power / noise_power
>>> decibels = 10 * math.log10(ratio)

>>> radians = 0.7
>>> height = math.sin(radians)
\end{verbatim}
\afterverb
%

첫 예제는 신호-대-소음비의 로그 밑이 10 을 계산한다.
수학 모듈이 자연 로그를 {\tt log} 함수를 호출해서 사용할 수 있도록 제공한다.

\index{로그 함수}
\index{함수!로그}
\index{사인 함수}
\index{라디안}
\index{삼각 함수}
\index{함수, 삼각법}
%\index{log function}
%\index{function!log}
%\index{sine function}
%\index{radian}
%\index{trigonometric function}
%\index{function, trigonometric}

두 번째 예제는 라디안의 사인값을 찾는 것이다. 
변수의 이름이 힌트를 주는데, {\tt sin}과 다른 삼각함수({\tt cos}, {\tt tan} 등)는 라디안을 인자로 받는다.
도(degree)에서 라디안(radian)으로 변환하기 위해서 360으로 나누고 $2\pi$를 곱한다.

\beforeverb
\begin{verbatim}
>>> degrees = 45
>>> radians = degrees / 360.0 * 2 * math.pi
>>> math.sin(radians)
0.707106781187
\end{verbatim}
\afterverb
%

{\tt math.pi} 표현식은 수학 모듈에서 {\tt pi} 변수를 얻는데, $\pi$ 값과 비교하여 15 자리수까지 정확하고 근사적으로 수렴한다.

\index{원주율}
%\index{pi}

삼각함수를 배웠다면, 앞선 연산 결과를 2에 루트를 씌우고 2로 나누어 비교 검증한다.

\index{sqrt 함수}
\index{함수!sqrt}
%\index{sqrt function}
%\index{function!sqrt}

\beforeverb
\begin{verbatim}
>>> math.sqrt(2) / 2.0
0.707106781187
\end{verbatim}
\afterverb
%

\section{신규 함수 추가}

지금까지 파이썬 설치 시 함께 있는 함수만 사용했지만 새로운 함수를 추가하는 것도 가능하다.
{\bf 함수 정의(function definition)}는 신규 함수명과 함수가 호출될 때 실행될 일련의 문장을 명세한다.
신규로 함수를 정의하면, 프로그램 실행 중에 반복해서 함수를 재사용할 수 있다. 

\index{함수}
\index{함수 정의}
\index{정의!함수}
%\index{function}
%\index{function definition}
%\index{definition!function}

다음에 예제가 있다.

\beforeverb
\begin{verbatim}
def print_lyrics():
    print "I'm a lumberjack, and I'm okay."
    print 'I sleep all night and I work all day.'
\end{verbatim}
\afterverb
%

{\tt def}는 "이것이 함수 정의다"를 표시하는 예약어다. 
함수 이름은 \verb"print_lyrics"이다.
함수 이름을 명명 규칙은 변수명과 동일하다. 
문자, 숫자, 그리고 몇몇 문장 부호는 사용할 수 있지만, 첫 문자가 숫자는 될 수 없다. 
함수 이름으로 예약어를 사용할 수 없고, 함수 이름과 동일한 변수명은 피해야 한다.

\index{def 예약어}
\index{예약어!def}
\index{인자}
%\index{def keyword}
%\index{keyword!def}
%\index{argument}

함수명 뒤 빈 괄호는 이 함수가 어떠한 인자도 갖지 않는다는 것을 나타낸다.
나중에, 입력값으로 인자를 가지는 함수를 작성해 볼 것이다.

\index{괄호!빈 괄호}
\index{머리 부분(헤더)}
\index{몸통 부문}
\index{들여쓰기}
\index{콜론}
%\index{parenthes론es!empty}
%\index{header}
%\index{body}
%\index{indentation}
%\index{colon}

함수 정의 첫번째 줄을 {\bf 머리 부문(헤더, header)}, 나머지 부문을 {\bf 몸통 부문(바디, body)}라고 부른다.
머리 부문은 콜론(:)으로 끝나고, 몸통 부문은 들여쓰기해야 한다.
파이썬 관례로 들여쓰기는 항상 4칸 공백이다. 
몸통 부문에는 제약 없이 문장을 작성할 수 있다.

print문의 문자열은 이중 인용부호로 감싼다. 
단일 인용부호나, 이중 인용부호나 차이는 없다.
대부분의 경우 단일 인용부호를 사용하고, 단일 인용부호가 문자열에 나타나는 경우, 이중 인용부호를 사용하여 단일 인용부호가 출력되게 감싼다.

\index{생략 부호}
%\index{ellipses}

만약 함수 정의를 인터렉티브 모드에서 타이핑을 하면, 
함수 정의가 끝나지 않았다는 것을 의미로 생략부호(...)가 출력된다.

\beforeverb
\begin{verbatim}
>>> def print_lyrics():
...     print "I'm a lumberjack, and I'm okay."
...     print 'I sleep all night and I work all day.'
...
\end{verbatim}
\afterverb
%

함수 정의를 끝내기 위해서 빈 줄을 입력한다. (스크립트에서는 반듯이 필요한 것은 아니다.) 

함수를 정의하게 되면 동일한 이름의 변수도 생성된다.

\beforeverb
\begin{verbatim}
>>> print print_lyrics
<function print_lyrics at 0xb7e99e9c>
>>> print type(print_lyrics)
<type 'function'>
\end{verbatim}
\afterverb
%

\verb"print_lyrics" 값은 \verb"'function'" 형을 가지는 {\bf 함수 객체(function object)}다. 

\index{함수 객체}
\index{객체!함수}
%\index{function object}
%\index{object!function}

신규 함수를 호출하는 구문은 내장 함수의 경우와 동일하다.

\beforeverb
\begin{verbatim}
>>> print_lyrics()
I'm a lumberjack, and I'm okay.
I sleep all night and I work all day.
\end{verbatim}
\afterverb
%

함수를 정의하면, 또 다른 함수 내부에서 사용이 가능하다.
예를 들어, 이전 후렴구를 반복하기 위해 \verb"repeat_lyrics" 함수를 작성할 수 있다.

\beforeverb
\begin{verbatim}
def repeat_lyrics():
    print_lyrics()
    print_lyrics()
\end{verbatim}
\afterverb
%

그리고 나서, \verb"repeat_lyrics" 함수를 호출한다.

\beforeverb
\begin{verbatim}
>>> repeat_lyrics()
I'm a lumberjack, and I'm okay.
I sleep all night and I work all day.
I'm a lumberjack, and I'm okay.
I sleep all night and I work all day.
\end{verbatim}
\afterverb
%

하지만, 이것이 실제 노래가 불려지는 것은 아니다.

\section{함수 정의와 사용법}
\index{함수 정의}
%\index{function definition}

앞 절의 코드 조각을 모아서 작성한 전체 프로그램은 다음과 같다.

\beforeverb
\begin{verbatim}
def print_lyrics():
    print "I'm a lumberjack, and I'm okay."
    print 'I sleep all night and I work all day.'

def repeat_lyrics():
    print_lyrics()
    print_lyrics()

repeat_lyrics()
\end{verbatim}
\afterverb
%

상기 프로그램에는 두개의 함수(\verb"print_lyrics", \verb"repeat_lyrics")가 있다.
함수 정의는 다른 문장처럼 수행되지만, 함수 객체를 생성한다는 점에서 차이가 있다.
함수 내부 문장은 함수가 호출되기 전까지 수행되지 않고, 함수 정의는 출력값도 생성하지 않는다.

\index{함수 정의 전 사용}
%\index{use before def}

예상하듯이, 함수를 실행하기 전에 함수를 생성해야 한다. 
다시 말해서, 처음으로 호출되기 전에 함수 정의가 실행되어야 한다.

\begin{ex}
상기 프로그램의 마지막 줄을 최상단으로 옮겨서 함수 정의 전에 호출되도록 프로그램을 고쳐보세요.
프로그램을 실행서 오류 메시지를 확인하세요.
\end{ex}

\begin{ex}
함수 호출을 맨 마지막으로 옮기고, \verb"repeat_lyrics" 함수 정의 뒤에 \verb"print_lyrics" 함수를 옮기세요.
프로그램을 실행하게 되면 무슨 일이 발생하나요?
\end{ex}


\section{실행 흐름}
\index{실행 흐름}
%\index{flow of execution}

처음으로 함수가 사용되기 전에 정의되었는지를 확인하기 위해서, 
명령문 실행 순서를 파악해야 하는데 이를 {\bf 실행 흐름(flow of execution)}이라고 한다.

프로그램 실행은 항상 프로그램 첫 문장부터 시작한다. 
명령문은 한번에 하나씩 위에서 아래로 실행된다.

함수 \emph{정의(definitions)}가 프로그램 실행 순서를 바꾸지는 않는다. 
하지만, 함수 내부의 문장은 함수가 호출될 때까지 실행이 되지 않는다는 것을 기억하자.

함수 호출은 프로그램 실행 흐름을 우회하는 것과 같다. 
다음 문장으로 가기 전에, 실행 흐름은 함수 몸통 부문을 실행하고는 건너 뛰기를 시작한 지점으로 다시 돌아온다.

함수가 또 다른 함수를 호출한다는 것을 기억할 때까지는 매우 간단하게 들린다.
함수 중간에서 프로그램이 또 다른 함수의 문장을 수행할지도 모른다. 
하지만, 새로운 함수를 실행하는 중간에 프로그램이 또 다른 함수를 실행할지도 모른다!

다행스럽게도, 파이썬은 프로그램 실행 위치를 정확히 추적한다. 
그래서, 함수가 실행을 완료할 때마다, 프로그램을 함수를 호출해서 떠난 지점으로 정확히 되돌려 놓는다. 
프로그램이 마지막에 도달했을 때, 프로그램은 종료한다.

이렇게 복잡한 이야기의 교훈은 무엇일까요? 
프로그램을 읽을 때, 위에서부터 아래로 읽을 필요는 없다. 
때때로, 실행 흐름을 따르는 것이 좀더 이치에 맞는다.

\section{매개 변수(parameter)와 인수(argument)}
\label{parameters}
\index{매개 변수}
\index{함수 매개 변수}
\index{인자}
\index{함수 인자}
%\index{parameter}
%\index{function parameter}
%\index{argument}
%\index{function argument}

지금까지 살펴본 몇몇 내장 함수는 인자를 요구한다. 
예를 들어, {\tt math.sin} 함수를 호출할 때, 숫자를 인자로 넘겨야 한다.
어떤 함수는 2개 이상의 인수를 받는다. {\tt math.pow} 는 밑과 지수 2개의 인자가 필요하다. 

인자는 함수 내부에서 {\bf 매개 변수(parameters)}로 불리는 변수로 할당된다.
하나의 인자를 받는 사용자 정의 함수(user-defined function)가 예제로 있다.

\index{괄호!매개변수 입력}
%\index{parentheses!parameters in}

\beforeverb
\begin{verbatim}
def print_twice(bruce):
    print bruce
    print bruce
\end{verbatim}
\afterverb
%

사용자 정의 함수는 인자를 받아 매개변수 {\tt bruce}에 할당한다. 
함수가 호출될 때, 매개변수의 값(무엇이든 관계 없이)을 두번 출력합니다.

사용자 정의 함수는 출력 가능한 임의의 값에 작동한다.

\beforeverb
\begin{verbatim}
>>> print_twice('Spam')
Spam
Spam
>>> print_twice(17)
17
17
>>> print_twice(math.pi)
3.14159265359
3.14159265359
\end{verbatim}
\afterverb
%

내장함수에 적용되는 동일한 구성 규칙이 사용자 정의 함수에도 적용되어서, \verb"print_twice" 함수 인자로 표현식 어떤 종류도 가능하다. 

\index{구성}
%\index{composition}

\beforeverb
\begin{verbatim}
>>> print_twice('Spam '*4)
Spam Spam Spam Spam
Spam Spam Spam Spam
>>> print_twice(math.cos(math.pi))
-1.0
-1.0
\end{verbatim}
\afterverb
%

함수가 호출되기 전에 인자에 대한 평가는 완료되어, 
예제에서 \verb"'Spam '*4"과 {\tt math.cos(math.pi)}은 단지 1회만 평가된다.

\index{인자}
%\index{argument}

변수도 인자로 사용이 가능하다.

\beforeverb
\begin{verbatim}
>>> michael = 'Eric, the half a bee.'
>>> print_twice(michael)
Eric, the half a bee.
Eric, the half a bee.
\end{verbatim}
\afterverb
%

인수자 넘기는 변수명({\tt michael})은 매개 변수명({\tt bruce})과 아무런 연관이 없다.
무슨 값이 호출된든지 호출하는 측과 상관이 없다. 
여기 \verb"print_twice" 함수에서는 누구나 {\tt bruce}라고 부르면 된다.

\section{결과있는 함수(fruitful function)와 빈 함수(void function)}

\index{결과있는 함수}
\index{빈 함수}
\index{함수, 결과있는}
\index{함수, 빈} 
%\index{fruitful function}
%\index{void function}
%\index{function, fruitful}
%\index{function, void} 

수학 함수와 같은 몇몇 함수는 결과를 만들어 낸다. 
좀더 좋은 이름이 없어서, 결과를 만들어 내는 함수를 {\bf 결과있는 함수(fruitful functions)}라고 명명한다.
\verb"print_twice"와 같이 액션을 수행하지만, 결과를 만들어 내지 않는 함수를 {\bf 빈 함수(void functions)}라고 부른다.

결과있는 함수를 호출할 때는 결과값을 가지고 뭔가를 하려고 한다. 
예를 들어, 결과값을 변수에 할당하거나, 표현식의 일부로 사용할 수 있다.

\beforeverb
\begin{verbatim}
x = math.cos(radians)
golden = (math.sqrt(5) + 1) / 2
\end{verbatim}
\afterverb
%

인터랙티브 모드에서 함수를 호출할 때, 파이썬은 결과를 화면에 출력한다.

\beforeverb
\begin{verbatim}
>>> math.sqrt(5)
2.2360679774997898
\end{verbatim}
\afterverb
%

하지만, 스크립트에서 결과있는 함수를 호출하고 변수에 결과값을 저장하지 않으면 반환되는 결과값은 안개속에 사라져간다!

\beforeverb
\begin{verbatim}
math.sqrt(5)
\end{verbatim}
\afterverb
%
이 스크립트는 5의 제곱근을 계산하지만, 변수에 결과값을 저장하거나, 화면에 출력하지 않아서 그다지 유용하지는 않다.

\index{인터랙티브 모드}
\index{스크립트 모드}
%\index{interactive mode}
%\index{script mode}

빈 함수(Void functions)는 화면에 출력하거나 무엇인가 다른 효과를 가지지만, 반환값이 없다.
빈 함수를 사용하여 결과에 변수를 할당하면, {\tt None}으로 불리는 특별한 값을 얻게 된다.

\index{None 특별값}
\index{특별값!None}
%\index{None special value}
%\index{special value!None}

\beforeverb
\begin{verbatim}
>>> result = print_twice('Bing')
Bing
Bing
>>> print result
None
\end{verbatim}
\afterverb
%

{\tt None} 값은 자신만의 특별한 값을 가지며, 문자열 \verb"'None'" 과는 같지 않다. 

\beforeverb
\begin{verbatim}
>>> print type(None)
<type 'NoneType'>
\end{verbatim}
\afterverb
%

함수에서 결과를 반환하기 위해서, 함수 내부에 {\tt return}문을 사용한다.
예를 들어, 두 숫자를 더해서 결과를 반환하는 {\tt addtwo}라는 간단한 함수를 작성할 수 있다. 

\beforeverb
\begin{verbatim}
def addtwo(a, b):
    added = a + b
    return added

x = addtwo(3, 5)
print x
\end{verbatim}
\afterverb
%

상기 스크립트가 실행될 때 print 문은 ``8''을 출력한다. 
왜냐하면, 3과 5를 인수로 받는 {\tt addtwo} 함수가 호출되기 때문이다.
함수 내부에 매개 변수 {\tt a}, {\tt b}는 각각 3, 5이다.
{\tt addtwo} 함수는 두 숫자 덧셈을 수행하고 {\tt added}라는 로컬 변수에 저장하고, {\tt return}문을 사용해서 덧셈 결과를 반환하고,
{\tt x} 라는 변수에 할당해서 출력한다.

\section{왜 함수를 사용하는가?}
\index{함수, 사용 이유}
%\index{function, reasons for}

프로그램을 함수로 나누는 고생을 할 가치가 있는지 명확하지 않을 수 있다. 다음에 여기 몇 가지 이유가 있다.

\begin{itemize}

\item 문장을 그룹으로 만들어 새로운 함수로 명명하는 것이 프로그램을 읽고, 이해하고, 디버그하기 좋게한다. 

\item 함수는 반복 코드를 제거해서 프로그램을 작고 콤팩트하게 만든다. 나중에 프로그램에 수정사항이 생기면, 단지 한 곳에서만 수정을 하면 된다.

\item 긴 프로그램을 함수로 나누어 작성하는 것은 작은 부분에서 버그를 수정할 수 있게 하고, 이를 조합해서 전체적으로 동작하는 프로그램을 만들 수 있다.

\item 잘 설계된 함수는 종종 많은 프로그램에서 유용하게 사용된다. 잘 설계된 프로그램을 작성하고 디버그를 해서 오류가 없이 만들게 되면, 나중에 재사용도 용이하다.

\end{itemize}

책의 나머지 부분에서 이 개념을 설명하는 함수 정의를 종종 사용한다. 
"리스트에서 가장 작은 값을 찾아내는 것"과 같이 아이디어를 적절하게 추상화하여 함수를 작성하는 것이 함수를 만들고 사용하는 기술의 일부가 된다. 
나중에, 리스트에서 가장 작은 값을 찾아내는 코드를 보여 줄 것입니다. 리스트를 인수로 받아 가장 작은 값을 
반환하는 {\tt min} 함수를 작성해서 여러분에게 보여드릴 것이다.

\section{디버깅}
\label{editor}
\index{디버깅}
%\index{debugging}

텍스트 편집기로 스크립트를 작성한다면 공백과 탭으로 몇번씩 문제에 봉착했을 것입니다. 
이런 문제를 피하는 가장 최선의 방식은 절대 탭을 사용하지 말고 공백(스페이스)를 사용하는 것이다. 
파이썬을 인식하는 대부분의 텍스트 편집기는 디폴트로 이런 기능을 지원하지만, 몇몇 텍스트 편집기는 이런 기능을 지원하지 않아 탭과 공백 문제를 야기한다.

\index{공백}
%\index{whitespace}

탭과 공백은 통상 눈에 보이지 않기 때문에 디버그를 어렵게 한다. 
자동으로 들여쓰기를 해주는 편집기를 프로그램 작성 시 사용한다.

프로그램을 실행하기 전에 저장하는 것을 잊지 마세요. 
몇몇 개발 환경은 자동저장 기능을 지원하지만 그렇지 않는 것도 있다.
이런 이유 때문에 텍스트 편집기에서 작성한 개발 프로그램과 실행운영하고 있는 프로그램이 같지 않을 수도 있다.

동일하고 잘못된 프로그램을 반복적으로 실행한다면, 디버깅은 오래 걸릴 수 있다.

작성하고 있는 코드와 실행하는 코드가 일치하는지 필히 확인하자. 
확신을 하지 못한다면, 프로그램의 첫줄에 \verb"print 'hello'" 을 넣어서 실행해 보자.
\verb"hello"를 보지 못한다면, 작성하고 있는 프로그램과 실행하고 있는 프로그램은 다른 것이다.


\section{용어정의}

\begin{description}

\item[알고리즘(algorithm):] 특정 범주의 문제를 해결하는 일반적인 프로세스
\index{알고리즘}
%\index{algorithm}

\item[인자(argument):] 함수가 호출될 때 함수에 제공되는 값. 이 값은 함수 내부에 상응하는 매개 변수에 할당된다.
\index{인자}
%\index{argument}

\item[몸통 부문(body):] 함수 정의 내부에 일련의 문장
\index{몸통 부문}
%\index{body}

\item[구성(composition):] 
좀더 큰 표현식의 일부분으로 표현식을 사용하거나, 좀더 큰 문장의 일부로서의 문장
\index{구성}
%\index{composition}

\item[결정론적(deterministic):] 동일한 입력값이 주어지고 실행될 때마다 동일한 행동을 하는 프로그램에 관련된 것.
\index{결정론적}
%\index{deterministic}

\item[점 표기법(dot notation):] 점과 함수명으로 모듈명을 명세함으로써 다른 모듈의 함수를 호출하는 구문.
\index{점 표기법}
%\index{dot notation}

\item[실행 흐름(flow of execution):] 프로그램 실행 동안 명령문이 실행되는 순서.
\index{실행 흐름}
%\index{flow of execution}

\item[결과있는 함수(fruitful function):] 반환값을 가지는 함수.
\index{결과있는 함수}
%\index{fruitful function}

\item[함수(function):] 유용한 연산을 수행하는 이름을 가진 일련의 명령문.
함수는 인수를 가질 수도 갖지 않을 수도 있고, 결과값을 생성할 수도 생성하지 않을 수도 있다.
\index{함수}
%\index{function}

\item[함수 호출(function call):] 함수를 실행하는 명령문. 함수 이름과 인자 리스트로 구성된다.
\index{함수 호출}
%\index{function call}

\item[함수 정의(function definition):] 신규 함수를 정의하는 명령문으로 이름, 매개변수, 실행 명령문을 명세한다.
\index{함수 정의}
%\index{function definition}

\item[함수 객체(function object):] 함수 정의로 생성되는 값. 함수명은 함수 객체를 참조하는 변수다.
\index{함수 객체}
%\index{function object}

\item[머리 부문(header):] 함수 정의의 첫번째 줄
\index{머리 부문}
%\index{header}

\item[가져오기 문(import statement):] 모듈 파일을 읽어 모듈 개체를 생성하는 명령문
\index{import 문장}
\index{문장!import}
%\index{import statement}
%\index{statement!import}

\item[모듈 개체(module object):] {\tt import}문에 의해서 생성된 모듈에 정의된 코드와 데이터에 접근할 수 있는 값
\index{모듈}
%\index{module}

\item[매개 변수(parameter):] 인자로 전달된 값을 참조하기 위해 함수 내부에 사용되는 이름
\index{매개 변수}
%\index{parameter}

\item[의사 난수(pseudorandom):] 난수처럼 보이는 일련의 숫자와 관련되어 있지만, 결정론적 프로그램에 의해 생성된다.
\index{의사 난수}
%\index{pseudorandom}

\item[반환 값(return value):] 함수의 결과. 함수 호출이 표현식으로 사용된다면, 반환값은 표현식의 값이 된다. 
\index{반환 값}
%\index{return value}

\item[빈 함수(void function):] 반환값을 갖지 않는 함수
\index{빈 함수}
%\index{void function}

\end{description}


\section{연습문제}

\begin{ex}
파이썬 "def" 키워드의 목적은 무엇입니까?

a) "다음의 코드는 정말 좋다"라는 의미를 가진 속어\\
b) 함수의 시작을 표시한다.\\
c) 다음의 들여쓰기 코드 부문은 나중을 위해 저장되야 된다는 것을 표시한다.\\
d) b와 c 모두 사실\\
e) 위 모두 거짓
\end{ex}

\begin{ex}

다음 파이썬 프로그램은 무엇을 출력할까요?

\beforeverb
\begin{verbatim}
def fred():
   print "Zap"

def jane():
   print "ABC"

jane()
fred()
jane()
\end{verbatim}
\afterverb
%
a) Zap ABC jane fred jane\\
b) Zap ABC Zap\\
c) ABC Zap jane\\
d) ABC Zap ABC\\
e) Zap Zap Zap
\end{ex}

\begin{ex}

프로그램 작성 시 ({\tt hours}과 {\tt rate})을 매개 변수로 갖는 함수 {\tt computepay}을 생성하여,
초과근무에 대해서는 50% 초과 근무수당을 지급하는 봉급 계산 프로그램을 다시 작성하세요.

\begin{verbatim}
Enter Hours: 45
Enter Rate: 10
Pay: 475.0
\end{verbatim}
\end{ex}

\begin{ex}

매개 변수로 점수를 받아 문자열로 등급을 반환하는 {\tt computegrade} 함수를 사용하여
앞장의 등급 프로그램을 다시 작성하세요.

\begin{verbatim}
Score   Grade
> 0.9     A
> 0.8     B
> 0.7     C
> 0.6     D
<= 0.6    F

Program Execution:

Enter score: 0.95
A

Enter score: perfect
Bad score

Enter score: 10.0
Bad score

Enter score: 0.75
C

Enter score: 0.5
F
\end{verbatim}

반복적으로 프로그램을 실행해서 다양한 다른 입력값을 테스트해 보세요.

\end{ex}


