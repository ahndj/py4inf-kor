% LaTeX source for ``Python for Informatics: Exploring Information''
% Copyright (c)  2010-  Charles R. Severance, All Rights Reserved
% 한국어 번역 : 이광춘, 한정수

\chapter{딕셔너리(Dictionaries)}

\index{딕셔너리 (dictionary)}
\index{자료형 (type)!dict}
\index{키 (key)}
\index{키-값 페어 (key-value pair)}
\index{인덱스 (index)}

{\bf 딕셔너리(dictionary)}는 리스트 같지만 좀더 일반적이다. 
리스트에서 위치(인텍스)는 정수이지만, 딕셔너리에서는 인덱스는 임의 자료형(type)이 될 수 있다.

딕셔너리를 인덱스 집합({\bf 키(keys)}라고 부름)에서 값(value) 집합으로 사상(mapping)하는 것으로 생각할 수 있다. 
각각의 키는 값에 대응한다. 
키와 값을 연관시키는 것을 {\bf 키-값 페어(key-value pair)}라고 부르고, 종종 항목(item)으로도 부른다.

한 예제로, 영어 단어에서 스페인 단어에 매핑되는 사전을 만들 것이다. 
키와 값은 모두 문자열이다.

{\tt dict} 함수는 항목이 전혀 없는 사전을 신규로 생성한다. 
{\tt dict}는 내장함수명이어서, 변수명으로 사용하는 것을 피해야 한다.

\index{dict 함수 (dict function)}
\index{함수 (function)!dict}

\beforeverb
\begin{verbatim}
>>> eng2sp = dict()
>>> print eng2sp
{}
\end{verbatim}
\afterverb

구불구불한 괄호 \verb"{}"는 빈 딕셔너리를 나타낸다. 
딕셔너리에 항목을 추가하기 위해서 꺾쇠 괄호를 사용한다.

\index{구불구불한 괄호 (squiggly bracket)}
\index{꺾쇠 괄호 (bracket)!구불구불 (squiggly)}

\beforeverb
\begin{verbatim}
>>> eng2sp['one'] = 'uno'
\end{verbatim}
\afterverb
%

상기 라인은 키 {\tt 'one'}에서 값 \verb"'uno'"를 매핑하는 항목을 생성한다. 
딕셔너리를 다시 출력하면, 키와 값 사이에 콜론(:)을 가진 키-값 페어(key-value pair)를 볼 수 있다.

\beforeverb
\begin{verbatim}
>>> print eng2sp
{'one': 'uno'}
\end{verbatim}
\afterverb
%

출력 형식이 또한 입력 형식이다. 
예를 들어, 세개 항목을 가진 신규 딕셔너리를 생성할 수 있다.

\beforeverb
\begin{verbatim}
>>> eng2sp = {'one': 'uno', 'two': 'dos', 'three': 'tres'}
\end{verbatim}
\afterverb
%

{\tt eng2sp}을 출력하면, 놀랄 것이다.

\beforeverb
\begin{verbatim}
>>> print eng2sp
{'one': 'uno', 'three': 'tres', 'two': 'dos'}
\end{verbatim}
\afterverb
%

키-값 페어(key-value pair) 순서가 같지 않다. 
사실 동일한 사례를 여러분의 컴퓨터에서 입력하면, 다른 결과를 얻게 된다.
일반적으로, 딕셔너리 항목 순서는 예측 가능하지 않다.

딕셔너리 요소가 정수 인덱스로 색인되지 않아서 문제되지는 않는다.
대신에, 키를 사용해서 상응하는 값을 찾을 수 있다.

\beforeverb
\begin{verbatim}
>>> print eng2sp['two']
'dos'
\end{verbatim}
\afterverb
%

{\tt 'two'} 키는 항상 값 \verb"'dos'"에 상응되어서 딕셔너리 항목 순서는 문제가 되지 않는다.

만약 키가 딕셔너리에 존재하지 않으면, 예외 오류가 발생한다.

\index{예외 (exception)!키 오류 (KeyError)}
\index{키 오류 (KeyError)}

\beforeverb
\begin{verbatim}
>>> print eng2sp['four']
KeyError: 'four'
\end{verbatim}
\afterverb
%

{\tt len} 함수를 딕셔너리에 사용해서, 키-값 페어(key-value pair) 항목 개수를 반환한다.

\index{len 함수 (len function)}
\index{함수 (function)!len}

\beforeverb
\begin{verbatim}
>>> len(eng2sp)
3
\end{verbatim}
\afterverb
%

{\tt in} 연산자도 딕셔너리에 작동되는데, 어떤 것이 딕셔너리 \emph{키(key)}에 있는지 알려준다. (값(value)으로 나타내는 것은 충분히 좋지는 않다.)

\index{소속 (membership)!딕셔너리 (dictionary)}
\index{in 연산자 (in operator)}
\index{연산자 (operator)!in}

\beforeverb
\begin{verbatim}
>>> 'one' in eng2sp
True
>>> 'uno' in eng2sp
False
\end{verbatim}
\afterverb
%

딕셔너리에 무엇이 값으로 있는지 확인하기 위해서, {\tt values} 메쏘드를 해서 리스트로 값을 반환받고 나서 {\tt in} 연산자를 사용하여 확인한다.  

\index{values 메쏘드 (values method)}
\index{메쏘드 (method)!값 (values)}

\beforeverb
\begin{verbatim}
>>> vals = eng2sp.values()
>>> 'uno' in vals
True
\end{verbatim}
\afterverb
%

{\tt in} 연산자는 리스트와 딕셔너리에 각기 다른 알고리즘을 사용한다. 
리스트에 대해서 선형 검색 알고리즘을 사용한다.
리스트가 길어짐에 따라 검색 시간은 리스트 길이에 비례하여 길어진다. 
딕셔너리에 대해서 파이썬은 {\bf 해쉬 테이블(hash table)}로 불리는 놀라운 특성을 가진 알고리즘을 사용한다. 
얼마나 많은 항목이 딕셔너리에 있는지에 관계없이 {\tt in} 연산자는 대략 동일한 시간이 소요된다.
왜 해쉬 함수가 마술 같은지에 대해서는 설명하지 않지만, \url{wikipedia.org/wiki/Hash_table} 에서 좀더 많은 정보를 얻을 수 있다. 

\index{해쉬 테이블 (hash table)}

\begin{ex}
\label{wordlist2}

\index{집합 소속 (set membership)}
\index{소속 (membership)!집합 (set)}

{\tt words.txt} 단어를 읽어서 딕셔너리에 키로 저장하는 프로그램을 작성하세요.
값이 무엇이든지 상관없습니다. 
딕셔너리에 문자열을 확인하는 가장 빠른 방법으로 {\tt in} 연산자를 사용할 수 있습니다.

\end{ex}


\section{계수기(counter) 집합으로서 딕셔너리}
\label{histogram}

\index{계수기 (counter)}

문자열이 주어진 상태에서, 각 문자가 얼마나 나타나는지를 센다고 가정합시다.
몇 가지 방법이 아래에 있습니다.

\begin{enumerate}

\item 26개 변수를 알파벳 문자 각각에 대해 생성한다. 
그리고 나서 아마도 연쇄 조건문을 사용하여 문자열을 훑고 해당 계수기(counter)를 하나씩 증가한다.

\item 26개 요소를 가진 리스트를 생성한다. 
내장함수 {\tt ord}를 사용해서 각 문자를 숫자로 변환한다. 
리스트 안에 인덱스로 숫자를 사용해서 적당한 계수기(counter)를 증가한다.

\item 키(key)로 문자, 계수기(counter)로 해당 값(value)을 갖는 딕셔너리를 생성한다. 
처음 문자를 만나면, 딕셔너리에 항목으로 추가한다.
추가한 후에는 기존 항목 값을 증가한다.

\end{enumerate}

상기 3개 옵션은 동일한 연산을 수행하지만, 각각은 다른 방식으로 연산을 구현한다.

\index{구현 (implementation)}

{\bf 구현(implementation)}은 연산(computation)을 수행하는 방법이다. 
어떤 구현 방법이 다른 것 보다 낫다.
예를 들어, 딕셔너리 구현의 장점은 사전에 문자열에서 어떤 문자가 나타날지 몰라도 된다. 다만 나타날 문자에 대한 공간만 준비하면 된다는 것이다.

다음에 딕셔너리로 구현한 코드가 있다.

\beforeverb
\begin{verbatim}
word = 'brontosaurus'
d = dict()
for c in word:
    if c not in d:
        d[c] = 1
    else:
        d[c] = d[c] + 1
print d
\end{verbatim}
\afterverb
%

계수기(counter) 혹은 빈도에 대한 통계 용어인 {\bf 히스토그램(histogram)}을 효과적으로 계산해보자.

\index{히스토그램 (histogram)}
\index{빈도 (frequency)}
\index{운행법 (traversal)}

{\tt for} 루프는 문자열을 훑는다. 매번 루프를 반복할 때마다 딕셔너리에 문자 {\tt c}가 없다면, 키 {\tt c}와 초기값 1을 가진 새로운 항목을 생성한다.
문자 {\tt c}가 이미 딕셔너리에 존재한다면, {\tt d[c]}을 증가한다.

\index{히스토그램 (histogram)}

다음 프로그램 실행 결과가 있다.

\beforeverb
\begin{verbatim}
{'a': 1, 'b': 1, 'o': 2, 'n': 1, 's': 2, 'r': 2, 'u': 2, 't': 1}
\end{verbatim}
\afterverb
%

히스토그램은 문자 {\tt 'a'}, \verb"'b'"는 1회, \verb"'o'"는 2회 등등 나타남을 보여준다.

\index{get 메쏘드 (get method)}
\index{메소뜨 (method)!get}

딕셔너리에는 키와 디폴트(default) 값을 갖는 {\tt get} 메쏘드가 있다. 
딕셔너리에 키가 나타나면, {\tt get} 메쏘드는 해당 값을 반환하고, 해당 값이 없으면 디폴트 값을 반환한다. 예를 들어,

\beforeverb
\begin{verbatim}
>>> counts = { 'chuck' : 1 , 'annie' : 42, 'jan': 100}
>>> print counts.get('jan', 0)
100
>>> print counts.get('tim', 0)
0
\end{verbatim}
\afterverb
%

{\tt get} 메쏘드를 사용해서 상기 히스토그램 루프를 좀더 간결하게 작성할 수 있다.
{\tt get} 메쏘드는 딕셔너리에 키가 존재하지 않는 경우를 자동적으로 다루기 때문에, {\tt if}문을 없애 4줄을 1줄로 줄일 수 있다.

\beforeverb
\begin{verbatim}
word = 'brontosaurus'
d = dict()
for c in word:
    d[c] = d.get(c,0) + 1
print d
\end{verbatim}
\afterverb
%

계수기(counter) 루프를 단순화하려고 {\tt get}메쏘드를 사용하는 것은 파이썬에서 흔히 사용되는 ''숙어(idiom)''가 되고, 책 후반부까지 많이 사용할 것이다.
시간을 가지고서 잠시 {\tt if} 문과 {\tt in} 연산자를 사용한 루프와 {\tt get}메쏘드를 사용한 루프를 비교해 보세요.
동일한 연산을 수행하지만, 하나가 더 간결한다.

\index{숙어 (idiom)}

\section{딕셔너리와 파일}

딕셔너리의 흔한 사용법 중의 하나는 텍스트로 작성된 파일에 단어 빈도를 세는 것이다. 
\url{http://shakespeare.mit.edu/Tragedy/romeoandjuliet/romeo_juliet.2.2.html} 사이트 덕분에 
\emph{로미오와 쥴리엣(Romeo and Juliet)} 텍스트 파일에서 시작합시다.

처음 연습으로 구두점이 없는 짧고 간략한 텍스트 버젼을 사용한다. 
나중에 구두점이 포함된 전체 텍스트로 작업할 것이다.

\beforeverb
\begin{verbatim}
But soft what light through yonder window breaks
It is the east and Juliet is the sun
Arise fair sun and kill the envious moon
Who is already sick and pale with grief
\end{verbatim}
\afterverb
%

파일 라인을 읽고, 각 라인을 단어 리스트로 쪼개고, 루프를 돌려 사전을 이용하여 각 단어의 빈도수를 세는 파이썬 프로그램을 작성한다.

\index{중첩 루프 (nested loops)}
\index{루프 (loop)!중첩 (nested)}

두 개의 {\tt for} 루프를 사용한다. 
외곽 루프는 파일 라인을 읽고, 내부 루프는 특정 라인의 단어 각각에 대해 반복한다.
하나의 루프는 \emph{외곽} 루프가 되고, 또 다른 루프는 \emph{내부} 루프가 되어서 {\bf 중첩 루프(nested loops)}라고 불리는 패턴 사례다.

외곽 루프가 한번 반복을 할 때마다 내부 루프는 모든 반복을 수행하기 때문에 내부 루프는 ''좀더 빨리'' 반복을 수행하고 외곽 루프는 
좀더 천천히 반복을 수행하는 것으로 생각할 수 있다.

\index{로미오와 쥴리엣 (Romeo and Juliet)}

두 중첩 루프의 조합이 입력 파일의 모든 라인에 있는 모든 단어의 빈도를 계수(count)하도록 보증합니다.

\beforeverb
\begin{verbatim}
fname = raw_input('Enter the file name: ')
try:
    fhand = open(fname)
except:
    print 'File cannot be opened:', fname
    exit()

counts = dict()
for line in fhand:
    words = line.split()
    for word in words:
        if word not in counts:
            counts[word] = 1
        else:
            counts[word] += 1

print counts
\end{verbatim}
\afterverb
%

프로그램을 실행하면, 정렬되지 않은 해쉬 순서로 모든 단어의 빈도수를 출력합니다.
{\tt romeo.txt} 파일은 \url{www.py4inf.com/code/romeo.txt}에서 다운로드 가능하다.

\beforeverb
\begin{verbatim}
python count1.py 
Enter the file name: romeo.txt
{'and': 3, 'envious': 1, 'already': 1, 'fair': 1, 
'is': 3, 'through': 1, 'pale': 1, 'yonder': 1, 
'what': 1, 'sun': 2, 'Who': 1, 'But': 1, 'moon': 1, 
'window': 1, 'sick': 1, 'east': 1, 'breaks': 1, 
'grief': 1, 'with': 1, 'light': 1, 'It': 1, 'Arise': 1, 
'kill': 1, 'the': 3, 'soft': 1, 'Juliet': 1}
\end{verbatim}
\afterverb
%

가장 높은 빈도 단어와 빈도수를 찾기 위해서 딕셔너리를 훑는 것이 불편하다. 
좀더 도움이 되는 출력결과를 만들려고 파이썬 코드를 추가하자.

\section{반복과 딕셔너리}

\index{딕셔너리 (dictionary)!루핑 (looping with)}
\index{루핑 (looping)!딕셔너리 (with dictionaries)}
\index{운행법 (traversal)}

{\tt for}문에 순서(sequence)로서 딕셔너리를 사용한다면, 딕셔너리 키를 훑는다. 
루프는 각 키와 해당 값을 출력한다.

\beforeverb
\begin{verbatim}
counts = { 'chuck' : 1 , 'annie' : 42, 'jan': 100}
for key in counts:
    print key, counts[key]
\end{verbatim}
\afterverb
%

출력은 다음과 같다.

\beforeverb
\begin{verbatim}
jan 100
chuck 1
annie 42
\end{verbatim}
\afterverb
%

다시 한번, 키는 특별한 순서가 없다.

\index{숙어 (idiom)}

이 패턴을 사용해서 앞서 기술한 다양한 루프 숙어를 구현한다.
예를 들어, 딕셔너리에서 10 보다 큰 값을 가진 항목을 모두 찾고자 한다면, 다음과 같이 코드를 작성한다.

\beforeverb
\begin{verbatim}
counts = { 'chuck' : 1 , 'annie' : 42, 'jan': 100}
for key in counts:
    if counts[key] > 10 :
        print key, counts[key]
\end{verbatim}
\afterverb
%

{\tt for} 루프는 딕셔너리 {\em 키(keys)}를 반복한다. 
그래서, 인덱스 연산자를 사용해서 각 키에 상응하는 {\em 값(value)}을 가져와야 한다.
여기 출력값이 있다.

\beforeverb
\begin{verbatim}
jan 100
annie 42
\end{verbatim}
\afterverb
%

10 이상 값만 가진 항목만 볼 수 있다.

\index{keys 메쏘드 (keys method)}
\index{메쏘드 (method)!keys}

알파벳 순으로 키를 출력하고자 한다면, 딕셔너리 객체의 {\tt keys} 메쏘드를 사용해서 딕셔너리 키 리스트를 생성한다.
그리고 나서 리스트를 정렬하고, 정렬된 리스트를 루프 돌리고, 아래와 같이 정렬된 순서로 키/값 페어(key/value pair)를 출력한다.

\beforeverb
\begin{verbatim}
counts = { 'chuck' : 1 , 'annie' : 42, 'jan': 100}
lst = counts.keys()
print lst
lst.sort()
for key in lst:
    print key, counts[key]
\end{verbatim}
\afterverb
%

다음에 출력결과가 있다.

\beforeverb
\begin{verbatim}
['jan', 'chuck', 'annie']
annie 42
chuck 1
jan 100
\end{verbatim}
\afterverb
%

처음에 {\tt keys} 메쏘드로부터 얻은 정렬되지 않은 키 리스트가 있고, 
그리고 나서 {\tt for} 루프로 정렬된 키/값 페어(key/value pair)가 있다.

\section{고급 텍스트 파싱}

\index{로미오와 쥴리엣 (Romeo and Juliet)}

{\tt romeo.txt} 파일을 사용한 상기 예제에서, 수작업으로 모든 구두점을 제거해서 가능한 단순하게 만들었다.
실제 텍스트는 아래 보여지는 것처럼 많은 구두점이 있다.

\beforeverb
\begin{verbatim}
But, soft! what light through yonder window breaks?
It is the east, and Juliet is the sun.
Arise, fair sun, and kill the envious moon,
Who is already sick and pale with grief,
\end{verbatim}
\afterverb
%

파이썬 {\tt split} 함수는 공백을 찾고 공백으로 구분되는 토큰으로 단어를 처리하기 때문에, 
``soft!'' 와 ``soft''는 다른 단어로 처리되고 각 단어에 대해서 별도 딕셔너리 항목을 생성한다.

파일에 대문자가 있어서, ``who''와 ``Who''를 다른 단어, 다른 빈도수를 가진 것으로 처리한다.

{\tt lower}, {\tt punctuation}, {\tt translate} 문자열 메쏘드를 사용해서 상기 문제를 해결할 수 있다.
{\tt translate} 메쏘드가 가장 적합하다. 
{\tt translate} 메쏘드에 대한 문서가 다음에 있다.

\verb"string.translate(s, table[, deletechars])"

\emph{(만약 존재한다면) deletechars 에 있는 모든 문자를 삭제한다.
그리고 나서 순서수(ordinal)로 색인된 각 문자를 번역하는 256-문자열 테이블(table)을 사용해서 문자를 번역한다.
만약 테이블이 None 이면, 문자 삭제 단계만 수행된다. }

{\tt table}을 명세하지는 않을 것이고, {\tt deletechars} 매개변수를 사용해서 모든 구두점을 삭제할 것이다.
파이썬이 ''구두점''으로 간주하는 문자 리스트를 출력하게 할 것이다.

\beforeverb
\begin{verbatim}
>>> import string
>>> string.punctuation
'!"#$%&\'()*+,-./:;<=>?@[\\]^_`{|}~'
\end{verbatim}
\afterverb
%

프로그램을 다음과 같은 수정을 한다.

\beforeverb
\begin{verbatim}
import string                                          # 신규 코드

fname = raw_input('Enter the file name: ')
try:
    fhand = open(fname)
except:
    print 'File cannot be opened:', fname
    exit()

counts = dict()
for line in fhand:
    line = line.translate(None, string.punctuation)    # 신규 코드
    line = line.lower()                                # 신규 코드
    words = line.split()
    for word in words:
        if word not in counts:
            counts[word] = 1
        else:
            counts[word] += 1

print counts
\end{verbatim}
\afterverb
%

{\tt translate} 메쏘드를 사용해서 모든 구두점을 제거했고, {\tt lower} 메쏘드를 사용해서 라인을 소문자로 수정했다.
나머지 프로그램은 변경된게 없다. 
파이썬 2.5 이전 버젼에는 {\tt translate} 메쏘드가 첫 매개변수로 {\tt None}을 받지 않아서
{\tt translate} 메쏘드를 호출하기 위해서 다음 코드를 사용하세요.

\beforeverb
\begin{verbatim}
print a.translate(string.maketrans(' ',' '), string.punctuation
\end{verbatim}
\afterverb
%

"파이썬 예술(Art of Python)'' 혹은 ``파이썬스럽게 생각하기(Thinking Pythonically)''를 배우는 일부분은
파이썬은 흔한 자료 분석 문제에 대해서 내장 기능을 가지고 있는 것을 깨닫는 것이다.
시간이 지남에 따라, 충분한 예제 코드를 보고 충분한 문서를 읽는다. 
작업을 편하게 할 수 있게 이미 다른 사람이 작성한 코드가 존재하는지를 살펴보기 위해서 어디를 찾아봐야 하는지도 알게 될 것이다.

다음은 출력결과의 축약 버젼이다.

\beforeverb
\begin{verbatim}
Enter the file name: romeo-full.txt
{'swearst': 1, 'all': 6, 'afeard': 1, 'leave': 2, 'these': 2, 
'kinsmen': 2, 'what': 11, 'thinkst': 1, 'love': 24, 'cloak': 1, 
a': 24, 'orchard': 2, 'light': 5, 'lovers': 2, 'romeo': 40, 
'maiden': 1, 'whiteupturned': 1, 'juliet': 32, 'gentleman': 1, 
'it': 22, 'leans': 1, 'canst': 1, 'having': 1, ...}
\end{verbatim}
\afterverb
%

출력결과는 여전히 다루기 힘들어 보입니다. 
파이썬을 사용해서 정확히 찾고자는 하는 것을 찾았으나 파이썬 {\bf 튜플(tuples)}에 대해서 학습할 필요성을 느껴진다.
튜플을 학습할 때, 다시 이 예제를 살펴볼 것이다.

\section{디버깅}
\index{디버깅 (debugging)}

점점 더 큰 데이터로 작업함에 따라, 수작업으로 데이터를 확인하거나 출력을 통해서 디버깅을 하는 것이 어려울 수 있다.
큰 데이터를 디버깅하는 몇가지 방법이 있다.

\begin{description}

\item[입력값을 줄여라(Scale down the input):]]
가능하면, 데이터 크기를 줄여라. 예를 들어, 프로그램이 텍스트 파일을 읽는다면, 첫 10줄로 시작하거나, 찾을 수 있는 작은 예제로 시작하라.
데이터 파일을 편집하거나, 프로그램을 수정해서 첫 {\tt n} 라인만 읽도록 프로그램을 변경하라.

오류가 있다면, {\tt n}을 줄여서 오류를 재현하는 가장 작은 값으로 만들어라.
그리고 나서, 오류를 찾고 수정해 나감에 따라 점진적으로 늘려나가라.

\item[요약값과 자료형을 확인하라(Check summaries and types):] 
전체 데이터를 출력하고 검증하는 대신에 데이터를 요약하여 출력하는 것을 생각하라: 예를 들어, 딕셔너리 항목의 숫자 혹은 리스트 숫자의 총계

실행 오류(runtime errors)의 일반적인 원인은 올바른 자료형(right type)이 아니기 때문이다. 
이런 종류의 오류를 디버깅하기 위해서, 종종 값의 자료형을 출력하는 것으로 충분하다.

\item[자가 진단 작성(Write self-checks):]

종종 오류를 자동적으로 검출하는 코드를 작성한다. 
예를 들어, 리스트 숫자의 평균을 계산한다면, 결과값은 리스트의 가장 큰 값보다 클 수 없고, 가장 작은 값보다 작을 수 없다는 것을 확인할 수 있다. 
''완전히 비상식적인'' 결과를 탐지하기 때문에 ''건전성 검사(sanity check)''라고 부른다.

\index{건전성 검사 (sanity check)}
\index{일치성 검사 (consistency check)}

또 다른 검사법은 두가지 다른 연산의 결과를 비교해서 일치하는지 살펴보는 것이다. 
''일치성 검사(consistency check)''라고 부른다.

\item[고급 출력(Pretty print the output):] 디버깅 출력결과를 서식화하는 것은 오류 발견을 용이하게 한다.

\end{description}

다시 한번, 발판(scaffolding)을 만드는데 들인 시간이 디버깅에 소비되는 시간을 줄일 수 있다.

\index{발판 (scaffolding)}

\section{용어정의}

\begin{description}

\item[딕셔너리(dictionary):] 키(key)에서 해당 값으로 매핑(mapping)
\index{딕셔너리 (dictionary)}

\item[해쉬테이블(hashtable):] 파이썬 딕셔너리를 구현하기 위해 사용된 알고리즘
\index{해쉬테이블 (hashtable)}

\item[해쉬 함수(hash function):] 키에 대한 위치를 계산하기 위해서 해쉬테이블에서 사용되는 함수
\index{해쉬 함수 (hash function)}

\item[히스토그램(histogram):] 계수기(counter) 집합.
\index{히스토그램 (histogram)}

\item[구현(implementation):] 연산(computation)을 수행하는 방법
\index{구현 (implementation)}

\item[항목(item):] 키-값 페어(key-value pair)에 대한 또 다른 이름.
\index{항목 (item)!딕셔너리 (dictionary)}

\item[키(key):] 키-값 페어(key-value pair)의 첫번째 부분으로 딕셔너리에 나타나는 객체.
\index{키 (key)}

\item[키-값 페어(key-value pair):] 키에서 값으로 매핑 표현.
\index{키-값 페어 (key-value pair)}

\item[룩업(lookup):] 키를 가지고 해당 값을 찾는 딕셔너리 연산.
\index{룩업 (lookup)}

\item[중첩 루프(nested loops):] 루프 ''내부''에 하나 혹은 그 이상의 루프가 있음. 
외곽 루프가 1회 실행될 때, 내부 루프는 전체 반복을 완료함.
\index{중첩 루프 (nested loops)}
\index{루프 (loop)!중첩 (nested)}

\item[값(value):] 키-값 페어(key-value pair)의 두번째 부분으로 딕셔너리에 나타나는 객체. 
앞에서 사용한 단어 ''값(value)'' 보다 더 구체적이다.
\index{값 (value)}

\end{description}

\section{연습문제}

\begin{ex}

커밋(commit)이 무슨 요일에 수행되었는지에 따라 전자우편 메세지를 구분하는 프로그램을 작성하세요.
''From''으로 시작하는 라인을 찾고, 3번째 단어를 찾아서 요일별 횟수를 계수(count)하여 저장하세요.
프로그램 끝에 딕셔너리 내용을 출력하세요. (순서는 문제가 되지 않습니다.)

\beforeverb
\begin{verbatim}
Sample Line:
From stephen.marquard@uct.ac.za Sat Jan  5 09:14:16 2008

Sample Execution:
python dow.py
Enter a file name: mbox-short.txt
{'Fri': 20, 'Thu': 6, 'Sat': 1}
\end{verbatim}
\afterverb
\end{ex}

\begin{ex}
전자우편 로그(log)를 읽고, 히스토그램을 생성하는 프로그램을 작성하세요.
딕셔너리를 사용해서 전자우편 주소별로 얼마나 많은 전자우편이 왔는지를 계수(count)하고 딕셔너리를 출력합니다.

\beforeverb
\begin{verbatim}
Enter file name: mbox-short.txt
{'gopal.ramasammycook@gmail.com': 1, 'louis@media.berkeley.edu': 3, 
'cwen@iupui.edu': 5, 'antranig@caret.cam.ac.uk': 1, 
'rjlowe@iupui.edu': 2, 'gsilver@umich.edu': 3, 
'david.horwitz@uct.ac.za': 4, 'wagnermr@iupui.edu': 1, 
'zqian@umich.edu': 4, 'stephen.marquard@uct.ac.za': 2, 
'ray@media.berkeley.edu': 1}
\end{verbatim}
\afterverb
\end{ex}

\begin{ex}

상기 프로그램에 누가 가장 많은 전자우편 메시지를 가졌는지 알아내는 코드를 추가하세요.

결국, 모든 데이터를 읽고, 딕셔너리를 생성한다. 
최대 루프(장~\ref{maximumloop} 참조)를 사용해서 딕셔너리를 훑어서 누가 가장 많은 전자우편 메시지를 갖는지, 
그리고 그 사람이 얼마나 많은 메시지를 갖는지 출력한다.

% (see Section~\ref{maximumloop})
%\ref{최대값과 최소값 루프}

\beforeverb
\begin{verbatim}
Enter a file name: mbox-short.txt
cwen@iupui.edu 5

Enter a file name: mbox.txt
zqian@umich.edu 195
\end{verbatim}
\afterverb
\end{ex}

\begin{ex}
다음 프로그램은 주소 대신에 도메인 이름을 기록한다. 
누가 메일을 보냈는지 대신(즉, 전체 전자우편 주소) 메시지가 어디에서부터 왔는지 출처를 기록한다.
프로그램 마지막에 딕셔너리 내용을 출력한다.

\beforeverb
\begin{verbatim}
python schoolcount.py
Enter a file name: mbox-short.txt
{'media.berkeley.edu': 4, 'uct.ac.za': 6, 'umich.edu': 7, 
'gmail.com': 1, 'caret.cam.ac.uk': 1, 'iupui.edu': 8}
\end{verbatim}
\afterverb
\end{ex}

