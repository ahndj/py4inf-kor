% LaTeX source for ``Python for Informatics: Exploring Information''
% Copyright (c)  2010-  Charles R. Severance, All Rights Reserved

\chapter{리스트}

\index{list}
\index{type!list}


\section{리스틑 열이다.}

Like a string, a {\bf list} is a sequence of values.  In a string, the
values are characters; in a list, they can be any type.  The values in
list are called {\bf elements} or sometimes {\bf items}.

\index{element}
\index{sequence}
\index{item}

There are several ways to create a new list; the simplest is to
enclose the elements in square brackets (\verb"[" and \verb"]"):

\beforeverb
\begin{verbatim}
[10, 20, 30, 40]
['crunchy frog', 'ram bladder', 'lark vomit']
\end{verbatim}
\afterverb
%
The first example is a list of four integers.  The second is a list of
three strings.  The elements of a list don't have to be the same type.
The following list contains a string, a float, an integer, and
(lo!) another list:

\beforeverb
\begin{verbatim}
['spam', 2.0, 5, [10, 20]]
\end{verbatim}
\afterverb
%
A list within another list is {\bf nested}.

\index{nested list}
\index{list!nested}

A list that contains no elements is
called an empty list; you can create one with empty
brackets, \verb"[]".

\index{empty list}
\index{list!empty}

As you might expect, you can assign list values to variables:

\beforeverb
\begin{verbatim}
>>> cheeses = ['Cheddar', 'Edam', 'Gouda']
>>> numbers = [17, 123]
>>> empty = []
>>> print cheeses, numbers, empty
['Cheddar', 'Edam', 'Gouda'] [17, 123] []
\end{verbatim}
\afterverb
%

\index{assignment}

\section{Lists are mutable}

\index{list!element}
\index{access}
\index{index}
\index{bracket operator}
\index{operator!bracket}

The syntax for accessing the elements of a list is the same as for
accessing the characters of a string---the bracket operator.  The
expression inside the brackets specifies the index.  Remember that the
indices start at 0:

\beforeverb
\begin{verbatim}
>>> print cheeses[0]
Cheddar
\end{verbatim}
\afterverb
%
Unlike strings, lists are mutable because you can change the order 
of items in a list or reassign an item in a list.  
When the bracket operator appears on the left side of an assignment, 
it identifies the element of the list that will be assigned.

\index{mutability}

\beforeverb
\begin{verbatim}
>>> numbers = [17, 123]
>>> numbers[1] = 5
>>> print numbers
[17, 5]
\end{verbatim}
\afterverb
%
The one-eth element of {\tt numbers}, which
used to be 123, is now 5.

\index{index!starting at zero}
\index{zero, index starting at}

You can think of a list as a relationship between indices and
elements.  This relationship is called a {\bf mapping}; each index
``maps to'' one of the elements.  

\index{item assignment}
\index{assignment!item}

List indices work the same way as string indices:

\begin{itemize}

\item Any integer expression can be used as an index.

\item If you try to read or write an element that does not exist, you
get an {\tt IndexError}.

\index{exception!IndexError}
\index{IndexError}

\item If an index has a negative value, it counts backward from the
end of the list.

\end{itemize}

\index{list!index}


\index{list!membership}
\index{membership!list}
\index{in operator}
\index{operator!in}

The {\tt in} operator also works on lists.

\beforeverb
\begin{verbatim}
>>> cheeses = ['Cheddar', 'Edam', 'Gouda']
>>> 'Edam' in cheeses
True
>>> 'Brie' in cheeses
False
\end{verbatim}
\afterverb


\section{Traversing a list}
\index{list!traversal}
\index{traversal!list}
\index{for loop}
\index{loop!for}
\index{statement!for}

The most common way to traverse the elements of a list is
with a {\tt for} loop.  The syntax is the same as for strings:

\beforeverb
\begin{verbatim}
for cheese in cheeses:
    print cheese
\end{verbatim}
\afterverb
%
This works well if you only need to read the elements of the
list.  But if you want to write or update the elements, you
need the indices.  A common way to do that is to combine
the functions {\tt range} and {\tt len}:

\index{looping!with indices}
\index{index!looping with}

\beforeverb
\begin{verbatim}
for i in range(len(numbers)):
    numbers[i] = numbers[i] * 2
\end{verbatim}
\afterverb
%
This loop traverses the list and updates each element.  {\tt len}
returns the number of elements in the list.  {\tt range} returns
a list of indices from 0 to $n-1$, where $n$ is the length of
the list.  Each time through the loop {\tt i} gets the index
of the next element.  The assignment statement in the body uses
{\tt i} to read the old value of the element and to assign the
new value.

\index{item update}
\index{update!item}

A {\tt for} loop over an empty list never executes the body:

\beforeverb
\begin{verbatim}
for x in empty:
    print 'This never happens.'
\end{verbatim}
\afterverb
%
Although a list can contain another list, the nested
list still counts as a single element.  The length of this list is
four:

\index{nested list}
\index{list!nested}

\beforeverb
\begin{verbatim}
['spam', 1, ['Brie', 'Roquefort', 'Pol le Veq'], [1, 2, 3]]
\end{verbatim}
\afterverb



\section{List operations}
\index{list!operation}

The {\tt +} operator concatenates lists:

\index{concatenation!list}
\index{list!concatenation}

\beforeverb
\begin{verbatim}
>>> a = [1, 2, 3]
>>> b = [4, 5, 6]
>>> c = a + b
>>> print c
[1, 2, 3, 4, 5, 6]
\end{verbatim}
\afterverb
%
Similarly, the {\tt *} operator repeats a list a given number of times:

\index{repetition!list}
\index{list!repetition}

\beforeverb
\begin{verbatim}
>>> [0] * 4
[0, 0, 0, 0]
>>> [1, 2, 3] * 3
[1, 2, 3, 1, 2, 3, 1, 2, 3]
\end{verbatim}
\afterverb
%
The first example repeats {\tt [0]} four times.  The second example
repeats the list {\tt [1, 2, 3]} three times.


\section{List slices}

\index{slice operator}
\index{operator!slice}
\index{index!slice}
\index{list!slice}
\index{slice!list}

The slice operator also works on lists:

\beforeverb
\begin{verbatim}
>>> t = ['a', 'b', 'c', 'd', 'e', 'f']
>>> t[1:3]
['b', 'c']
>>> t[:4]
['a', 'b', 'c', 'd']
>>> t[3:]
['d', 'e', 'f']
\end{verbatim}
\afterverb
%
If you omit the first index, the slice starts at the beginning.
If you omit the second, the slice goes to the end.  So if you
omit both, the slice is a copy of the whole list.

\index{list!copy}
\index{slice!copy}
\index{copy!slice}

\beforeverb
\begin{verbatim}
>>> t[:]
['a', 'b', 'c', 'd', 'e', 'f']
\end{verbatim}
\afterverb
%
Since lists are mutable, it is often useful to make a copy
before performing operations that fold, spindle or mutilate
lists.

\index{mutability}

A slice operator on the left side of an assignment
can update multiple elements:

\index{slice!update}
\index{update!slice}

\beforeverb
\begin{verbatim}
>>> t = ['a', 'b', 'c', 'd', 'e', 'f']
>>> t[1:3] = ['x', 'y']
>>> print t
['a', 'x', 'y', 'd', 'e', 'f']
\end{verbatim}
\afterverb
%

\section{List methods}

\index{list!method}
\index{method, list}

Python provides methods that operate on lists.  For example,
{\tt append} adds a new element to the end of a list:

\index{append method}
\index{method!append}

\beforeverb
\begin{verbatim}
>>> t = ['a', 'b', 'c']
>>> t.append('d')
>>> print t
['a', 'b', 'c', 'd']
\end{verbatim}
\afterverb
%
{\tt extend} takes a list as an argument and appends all of
the elements:

\index{extend method}
\index{method!extend}

\beforeverb
\begin{verbatim}
>>> t1 = ['a', 'b', 'c']
>>> t2 = ['d', 'e']
>>> t1.extend(t2)
>>> print t1
['a', 'b', 'c', 'd', 'e']
\end{verbatim}
\afterverb
%
This example leaves {\tt t2} unmodified.

{\tt sort} arranges the elements of the list from low to high:

\index{sort method}
\index{method!sort}

\beforeverb
\begin{verbatim}
>>> t = ['d', 'c', 'e', 'b', 'a']
>>> t.sort()
>>> print t
['a', 'b', 'c', 'd', 'e']
\end{verbatim}
\afterverb
%
Most list methods are void; they modify the list and return {\tt None}.
If you accidentally write {\tt t = t.sort()}, you will be disappointed
with the result.

\index{void method}
\index{method!void}
\index{None special value}
\index{special value!None}

\section{Deleting elements}

\index{element deletion}
\index{deletion, element of list}

There are several ways to delete elements from a list.  If you
know the index of the element you want, you can use
{\tt pop}:

\index{pop method}
\index{method!pop}

\beforeverb
\begin{verbatim}
>>> t = ['a', 'b', 'c']
>>> x = t.pop(1)
>>> print t
['a', 'c']
>>> print x
b
\end{verbatim}
\afterverb
%
{\tt pop} modifies the list and returns the element that was removed.
If you don't provide an index, it deletes and returns the
last element.

If you don't need the removed value, you can use the {\tt del}
operator:

\index{del operator}
\index{operator!del}

\beforeverb
\begin{verbatim}
>>> t = ['a', 'b', 'c']
>>> del t[1]
>>> print t
['a', 'c']
\end{verbatim}
\afterverb
%

If you know the element you want to remove (but not the index), you
can use {\tt remove}:

\index{remove method}
\index{method!remove}

\beforeverb
\begin{verbatim}
>>> t = ['a', 'b', 'c']
>>> t.remove('b')
>>> print t
['a', 'c']
\end{verbatim}
\afterverb
%
The return value from {\tt remove} is {\tt None}.

\index{None special value}
\index{special value!None}

To remove more than one element, you can use {\tt del} with
a slice index:

\beforeverb
\begin{verbatim}
>>> t = ['a', 'b', 'c', 'd', 'e', 'f']
>>> del t[1:5]
>>> print t
['a', 'f']
\end{verbatim}
\afterverb
%
As usual, the slice selects all the elements up to, but not
including, the second index.

\section{Lists and functions}

There are a number of built-in functions that can be used on lists
that allow you to quickly look through a list without
writing your own loops:

\beforeverb
\begin{verbatim}
>>> nums = [3, 41, 12, 9, 74, 15]
>>> print len(nums)
6
>>> print max(nums)
74
>>> print min(nums)
3
>>> print sum(nums)
154
>>> print sum(nums)/len(nums)
25
\end{verbatim}
\afterverb
%
The {\tt sum()} function only works when the list elements are numbers.
The other functions ({\tt max()}, {\tt len()}, etc.) work with lists of
strings and other types that can be comparable.

We could rewrite an earlier program that computed the average of 
a list of numbers entered by the user using a list.

First, the program to compute an average without a list:

\beforeverb
\begin{verbatim}
total = 0
count = 0
while ( True ) :
    inp = raw_input('Enter a number: ')
    if inp == 'done' : break
    value = float(inp)
    total = total + value
    count = count + 1

average = total / count
print 'Average:', average
\end{verbatim}
\afterverb
%
In this program, we have {\tt count} and {\tt sum} variables to 
keep the number and running total of the user's numbers as 
we repeatedly prompt the user for a number.

We could simply remember each number as the user entered it 
and use built-in functions to compute the sum and count at
the end.

\beforeverb
\begin{verbatim}
numlist = list()
while ( True ) :
    inp = raw_input('Enter a number: ')
    if inp == 'done' : break
    value = float(inp)
    numlist.append(value)

average = sum(numlist) / len(numlist)
print 'Average:', average
\end{verbatim}
\afterverb
%
We make an empty list before the loop starts, and then each time we have 
a number, we append it to the list.  At the end of
the program, we simply compute the sum of the numbers in the 
list and divide it by the count of the numbers in the
list to come up with the average.

\section{Lists and strings}

\index{list}
\index{string}
\index{sequence}

A string is a sequence of characters and a list is a sequence
of values, but a list of characters is not the same as a
string.  To convert from a string to a list of characters,
you can use {\tt list}:

\index{list!function}
\index{function!list}

\beforeverb
\begin{verbatim}
>>> s = 'spam'
>>> t = list(s)
>>> print t
['s', 'p', 'a', 'm']
\end{verbatim}
\afterverb
%
Because {\tt list} is the name of a built-in function, you should
avoid using it as a variable name.  I also avoid {\tt l} because
it looks too much like {\tt 1}.  So that's why I use {\tt t}.

The {\tt list} function breaks a string into individual letters.  If
you want to break a string into words, you can use the {\tt split}
method:

\index{split method}
\index{method!split}

\beforeverb
\begin{verbatim}
>>> s = 'pining for the fjords'
>>> t = s.split()
>>> print t
['pining', 'for', 'the', 'fjords']
>>> print t[2]
the
\end{verbatim}
\afterverb
%
Once you have used {\tt split} to break the string into 
a list of tokens, you can use the index operator (square
bracket) to look at a particular word in the list.

You can call {\tt split} with 
an optional argument called a {\bf delimiter} specifies which
characters to use as word boundaries.
The following example
uses a hyphen as a delimiter:

\index{optional argument}
\index{argument!optional}
\index{delimiter}

\beforeverb
\begin{verbatim}
>>> s = 'spam-spam-spam'
>>> delimiter = '-'
>>> s.split(delimiter)
['spam', 'spam', 'spam']
\end{verbatim}
\afterverb
%
{\tt join} is the inverse of {\tt split}.  It
takes a list of strings and
concatenates the elements.  {\tt join} is a string method,
so you have to invoke it on the delimiter and pass the
list as a parameter:

\index{join method}
\index{method!join}
\index{concatenation}

\beforeverb
\begin{verbatim}
>>> t = ['pining', 'for', 'the', 'fjords']
>>> delimiter = ' '
>>> delimiter.join(t)
'pining for the fjords'
\end{verbatim}
\afterverb
%
In this case the delimiter is a space character, so
{\tt join} puts a space between words.  To concatenate
strings without spaces, you can use the empty string,
\verb"''", as a delimiter. 

\index{empty string}
\index{string!empty}


\section{Parsing lines}

Usually when we are reading a file 
we want to do something to the lines other than just 
printing the whole line.  Often we want to find the ``interesting
lines'' and then {\bf parse} the line to find some interesting
\emph{part} of the line.  What if we wanted to print out the day of the 
week from those lines that start with ``From ''.

\beforeverb
\begin{alltt}
From stephen.marquard@uct.ac.za {\bf Sat} Jan  5 09:14:16 2008
\end{alltt}
\afterverb

The {\tt split} method is very effective when faced with this 
kind of problem.
We can write a small program that looks for lines where the 
line starts with ``From '' and then {\tt split} those lines 
and then print out the third word in the line:

\beforeverb
\begin{verbatim}
fhand = open('mbox-short.txt')
for line in fhand:
    line = line.rstrip()
    if not line.startswith('From ') : continue
    words = line.split()
    print words[2]
\end{verbatim}
\afterverb
%
We also use the contracted form of the {\tt if}
statement where we put the {\tt continue } on the
same line as the {\tt if}.  This contracted form
of the {\tt if} functions the same as if the
{\tt continue} were on the next line and indented.

The program produces the following output:

\beforeverb
\begin{verbatim}
Sat
Fri
Fri
Fri
    ...
\end{verbatim}
\afterverb
%
Later, we will learn increasingly sophisticated techniques for
picking the lines to work on and how we pull those lines apart
to find the exact bit of information we are looking for.

\section{Objects and values}

\index{object}
\index{value}

If we execute these assignment statements:

\beforeverb
\begin{verbatim}
a = 'banana'
b = 'banana'
\end{verbatim}
\afterverb
%
We know that {\tt a} and {\tt b} both refer to a
string, but we don't
know whether they refer to the \emph{same} string.
There are two possible states:

\index{aliasing}

\beforefig
\centerline{\includegraphics{figs2/list1.eps}}
\afterfig

In one case, {\tt a} and {\tt b} refer to two different objects that
have the same value.  In the second case, they refer to the same
object.

\index{is operator}
\index{operator!is}

To check whether two variables refer to the same object, you can
use the {\tt is} operator.

\beforeverb
\begin{verbatim}
>>> a = 'banana'
>>> b = 'banana'
>>> a is b
True
\end{verbatim}
\afterverb
%
In this example, Python only created one string object,
and both {\tt a} and {\tt b} refer to it.

But when you create two lists, you get two objects:

\beforeverb
\begin{verbatim}
>>> a = [1, 2, 3]
>>> b = [1, 2, 3]
>>> a is b
False
\end{verbatim}
\afterverb
%

In this case we would say that the two lists are {\bf equivalent},
because they have the same elements, but not {\bf identical}, because
they are not the same object.  If two objects are identical, they are
also equivalent, but if they are equivalent, they are not necessarily
identical.

\index{equivalence}
\index{identity}

Until now, we have been using ``object'' and ``value''
interchangeably, but it is more precise to say that an object has a
value.  If you execute {\tt a = [1,2,3]}, {\tt a} refers to a list
object whose value is a particular sequence of elements.  If another
list has the same elements, we would say it has the same value.

\index{object}
\index{value}


\section{Aliasing}

\index{aliasing}
\index{reference!aliasing}

If {\tt a} refers to an object and you assign {\tt b = a},
then both variables refer to the same object:

\beforeverb
\begin{verbatim}
>>> a = [1, 2, 3]
>>> b = a
>>> b is a
True
\end{verbatim}
\afterverb
%

The association of a variable with an object is called a {\bf
reference}.  In this example, there are two references to the same
object.

\index{reference}

An object with more than one reference has more
than one name, so we say that the object is {\bf aliased}.

\index{mutability}

If the aliased object is mutable, 
changes made with one alias affect
the other:

\beforeverb
\begin{verbatim}
>>> b[0] = 17
>>> print a
[17, 2, 3]
\end{verbatim}
\afterverb
%
Although this behavior can be useful, it is error-prone.  In general,
it is safer to avoid aliasing when you are working with mutable
objects.

\index{immutability}

For immutable objects like strings, aliasing is not as much of a
problem.  In this example:

\beforeverb
\begin{verbatim}
a = 'banana'
b = 'banana'
\end{verbatim}
\afterverb
%
It almost never makes a difference whether {\tt a} and {\tt b} refer
to the same string or not.


\section{List arguments}

\index{list!as argument}
\index{argument}
\index{argument!list}
\index{reference}
\index{parameter}

When you pass a list to a function, the function gets a reference
to the list.
If the function modifies a list parameter, the caller sees the change.
For example, \verb"delete_head" removes the first element from a list:

\beforeverb
\begin{verbatim}
def delete_head(t):
    del t[0]
\end{verbatim}
\afterverb
%
Here's how it is used:

\beforeverb
\begin{verbatim}
>>> letters = ['a', 'b', 'c']
>>> delete_head(letters)
>>> print letters
['b', 'c']
\end{verbatim}
\afterverb
%
The parameter {\tt t} and the variable {\tt letters} are
aliases for the same object.  

It is important to distinguish between operations that
modify lists and operations that create new lists.  For
example, the {\tt append} method modifies a list, but the
{\tt +} operator creates a new list:

\index{append method}
\index{method!append}
\index{list!concatenation}
\index{concatenation!list}

\beforeverb
\begin{verbatim}
>>> t1 = [1, 2]
>>> t2 = t1.append(3)
>>> print t1
[1, 2, 3]
>>> print t2
None

>>> t3 = t1 + [3]
>>> print t3
[1, 2, 3]
>>> t2 is t3
False
\end{verbatim}
\afterverb

This difference is important when you write functions that
are supposed to modify lists.  For example, this function
\emph{does not} delete the head of a list:

\beforeverb
\begin{verbatim}
def bad_delete_head(t):
    t = t[1:]              # WRONG!
\end{verbatim}
\afterverb

The slice operator creates a new list and the assignment
makes {\tt t} refer to it, but none of that has any effect
on the list that was passed as an argument.

\index{slice operator}
\index{operator!slice}

An alternative is to write a function that creates and
returns a new list.  For
example, {\tt tail} returns all but the first
element of a list:

\beforeverb
\begin{verbatim}
def tail(t):
    return t[1:]
\end{verbatim}
\afterverb
%
This function leaves the original list unmodified.
Here's how it is used:

\beforeverb
\begin{verbatim}
>>> letters = ['a', 'b', 'c']
>>> rest = tail(letters)
>>> print rest
['b', 'c']
\end{verbatim}
\afterverb


\begin{ex}

Write a function called {\tt chop} that takes a list and modifies
it, removing the first and last elements, and returns {\tt None}.

Then write a function called {\tt middle} that takes a list and
returns a new list that contains all but the first and last
elements.

\end{ex}


\section{Debugging}
\index{debugging}

Careless use of lists (and other mutable objects)
can lead to long hours of debugging.  Here are some common
pitfalls and ways to avoid them:

\begin{enumerate}

\item Don't forget that most list methods modify the argument and
  return {\tt None}.  This is the opposite of the string methods,
  which return a new string and leave the original alone.

If you are used to writing string code like this:

\beforeverb
\begin{verbatim}
word = word.strip()
\end{verbatim}
\afterverb

It is tempting to write list code like this:

\beforeverb
\begin{verbatim}
t = t.sort()           # WRONG!
\end{verbatim}
\afterverb

\index{sort method}
\index{method!sort}

Because {\tt sort} returns {\tt None}, the
next operation you perform with {\tt t} is likely to fail.

Before using list methods and operators, you should read the
documentation carefully and then test them in interactive mode.  The
methods and operators that lists share with other sequences (like
strings) are documented at
\url{docs.python.org/lib/typesseq.html}.  The
methods and operators that only apply to mutable sequences
are documented at \url{docs.python.org/lib/typesseq-mutable.html}.


\item Pick an idiom and stick with it.
\index{idiom}

Part of the problem with lists is that there are too many
ways to do things.  For example, to remove an element from
a list, you can use {\tt pop}, {\tt remove}, {\tt del},
or even a slice assignment.

To add an element, you can use the {\tt append} method or
the {\tt +} operator.  But don't forget that these are right: 

\beforeverb
\begin{verbatim}
t.append(x)
t = t + [x]
\end{verbatim}
\afterverb

And these are wrong:

\beforeverb
\begin{verbatim}
t.append([x])          # WRONG!
t = t.append(x)        # WRONG!
t + [x]                # WRONG!
t = t + x              # WRONG!
\end{verbatim}
\afterverb

Try out each of these examples in interactive mode to make sure
you understand what they do.  Notice that only the last
one causes a runtime error; the other three are legal, but they
do the wrong thing.


\item Make copies to avoid aliasing.

\index{aliasing!copying to avoid}
\index{copy!to avoid aliasing}

If you want to use a method like {\tt sort} that modifies
the argument, but you need to keep the original list as
well, you can make a copy.

\beforeverb
\begin{verbatim}
orig = t[:]
t.sort()
\end{verbatim}
\afterverb

In this example you could also use the built-in function {\tt sorted},
which returns a new, sorted list and leaves the original alone.
But in that case you should avoid using {\tt sorted} as a variable
name!

\item Lists, {\tt split}, and files

When we read and parse files, there are many opportunities
to encounter input that can crash our program so it is a good 
idea to revisit the {\bf guardian} pattern when it comes
writing programs that read through a file 
and look for a ``needle in the haystack''.

Let's revisit our program that is looking for the day of the
week on the from lines of our file:

\beforeverb
\begin{alltt}
From stephen.marquard@uct.ac.za {\bf Sat} Jan  5 09:14:16 2008
\end{alltt}
\afterverb

Since we are breaking this line into words, we could dispense
with the use of {\tt startswith} and simply look at the 
first word of the line to determine if we are interested
in the line at all.  We can use {\tt continue} to skip lines
that don't have ``From'' as the first word as follows:

\beforeverb
\begin{verbatim}
fhand = open('mbox-short.txt')
for line in fhand:
    words = line.split()
    if words[0] != 'From' : continue
    print words[2]
\end{verbatim}
\afterverb
%
This looks much simpler and we don't even need to do the 
{\tt rstrip} to remove the newline at the end of the file.
But is it better?

\beforeverb
\begin{verbatim}
python search8.py 
Sat
Traceback (most recent call last):
  File "search8.py", line 5, in <module>
    if words[0] != 'From' : continue
IndexError: list index out of range
\end{verbatim}
\afterverb
%
It kind of works and we see the day from the first line
(Sat) but then the program fails with a traceback error.
What went wrong?  What messed-up data caused our elegant, 
clever and very Pythonic program to fail?

You could stare at it for a long time and puzzle through
it or ask someone for help, but the quicker and smarter
approach is to add a {\tt print} statement.  The best place
to add the print statement is right before the line where
the program failed and print out the data that seems to be causing
the failure.

Now this approach may generate a lot of lines of output but at 
least you will immediately have some clue as to the 
problem at hand.  So we add a print of the variable
{\tt words} right before line five.  We even 
add a prefix ``Debug:'' to the line so we can keep
our regular output separate from our debug output.

\beforeverb
\begin{verbatim}
for line in fhand:
    words = line.split()
    print 'Debug:', words
    if words[0] != 'From' : continue
    print words[2]
\end{verbatim}
\afterverb
%
When we run the program, a lot of output scrolls off the screen
but at the end, we see our debug output and the traceback so 
we know what happened just before the traceback.

\beforeverb
\begin{verbatim}
Debug: ['X-DSPAM-Confidence:', '0.8475']
Debug: ['X-DSPAM-Probability:', '0.0000']
Debug: []
Traceback (most recent call last):
  File "search9.py", line 6, in <module>
    if words[0] != 'From' : continue
IndexError: list index out of range
\end{verbatim}
\afterverb
%
Each debug line is printing the list of words which we get
when we {\tt split} the line into words.  When the program fails
the list of words is empty \verb"[]".  If we open the file in a text editor
and look at the file, at that point it looks as follows:

\beforeverb
\begin{verbatim}
X-DSPAM-Result: Innocent
X-DSPAM-Processed: Sat Jan  5 09:14:16 2008
X-DSPAM-Confidence: 0.8475
X-DSPAM-Probability: 0.0000

Details: http://source.sakaiproject.org/viewsvn/?view=rev&rev=39772
\end{verbatim}
\afterverb
%
The error occurs when our program encounters a blank line! Of course there
are ``zero words'' on a blank line.  Why didn't we think of that 
when we were writing the code.  When the code looks for the first
word (\verb"word[0]") to check to see if it matches ``From'', 
we get an ``index out of range'' error.

This of course is the perfect place to add some {\bf guardian} code 
to avoid checking the first word if the first word is not there.
There are many ways to protect this code, we will choose to 
check the number of words we have before we look at the first word:

\beforeverb
\begin{verbatim}
fhand = open('mbox-short.txt')
count = 0
for line in fhand:
    words = line.split()
    # print 'Debug:', words
    if len(words) == 0 : continue
    if words[0] != 'From' : continue
    print words[2]
\end{verbatim}
\afterverb
%
First we commented out the debug print statement instead of removing it 
in case our modification fails and we need to debug again.  Then we added
a guardian statement that checks to see if we have zero words, and if so, 
we use {\tt continue} to skip to the next line in the file.

We can think of the two {\tt continue} statements as helping us refine
the set of lines which are ``interesting'' to us and which we want 
to process some more.  A line which has no words is ``uninteresting'' to 
us so we skip to the next line.  A line which does not have ``From''
as its first word is uninteresting to us so we skip it.

The program as modified runs successfully so perhaps it is correct.  Our
guardian statement does make sure that the {\tt words[0]} will never fail, 
but perhaps it is not enough.  When we are programming, we must always be 
thinking, ``What might go wrong?''.

\begin{ex}
Figure out which line of the above program is still not properly guarded.
See if you can construct a text file which causes the program to fail
and then modify the program so that the line is properly guarded and 
test it to make sure it handles your new text file.
\end{ex}

\begin{ex}
Rewrite the guardian code in the above example without two
{\tt if} statements.  Instead use a compound logical expression using the
{\tt and} logical operator with a single {\tt if} statement.
\end{ex}


\end{enumerate}



\section{Glossary}

\begin{description}

\item[aliasing:] A circumstance where two or more variables refer to the same
object.
\index{aliasing}

\item[delimiter:] A character or string used to indicate where a
string should be split.
\index{delimiter}

\item[element:] One of the values in a list (or other sequence),
also called items.
\index{element}

\item[equivalent:] Having the same value.
\index{equivalent}

\item[index:] An integer value that indicates an element in a list.
\index{index}

\item[identical:] Being the same object (which implies equivalence).
\index{identical}

\item[list:] A sequence of values.
\index{list}

\item[list traversal:] The sequential accessing of each element in a list.
\index{list!traversal}

\item[nested list:] A list that is an element of another list.
\index{nested list}

\item[object:] Something a variable can refer to.  An object
has a type and a value.
\index{object}

\item[reference:] The association between a variable and its value.
\index{reference}

\end{description}


\section{Exercises}

\begin{ex}
Download a copy of the file from 
\url{www.py4inf.com/code/romeo.txt}
\index{Romeo and Juliet}

Write a program to open the file {\tt romeo.txt} and read it
line by line.  For each line, split the line into  a list of 
words using the {\tt split} function.

For each word, check to see if the word is already in a list.  
If the word is not in the list, add it to the list.  

When the program completes, sort and print the resulting words
in alphabetical order.

\begin{verbatim}
Enter file: romeo.txt
['Arise', 'But', 'It', 'Juliet', 'Who', 'already', 
'and', 'breaks', 'east', 'envious', 'fair', 'grief', 
'is', 'kill', 'light', 'moon', 'pale', 'sick', 'soft', 
'sun', 'the', 'through', 'what', 'window', 
'with', 'yonder']
\end{verbatim}
\end{ex}

\begin{ex}
Write a program to read through the mail box data and when you find 
line that starts with ``From'', you will split the line into 
words using the {\tt split} function. We are interested in 
who sent the message which is the second word on the From line.

{\tt From stephen.marquard@uct.ac.za Sat Jan  5 09:14:16 2008 }

You will parse the From line and print out the second word for 
each From line and then you will also count the number of 
From (not From:) lines and print out a count at the end.

This is a good sample output with a few lines removed:

\beforeverb
\begin{verbatim}
python fromcount.py 
Enter a file name: mbox-short.txt
stephen.marquard@uct.ac.za
louis@media.berkeley.edu
zqian@umich.edu

[...some output removed...]

ray@media.berkeley.edu
cwen@iupui.edu
cwen@iupui.edu
cwen@iupui.edu
There were 27 lines in the file with From as the first word
\end{verbatim}
\afterverb
%
\end{ex}

\begin{ex}
Rewrite the program that prompts the user for a list of 
numbers and prints out the maximum and minimum of the
numbers at the end when the user enters ``done''.  Write
the program to store the numbers the user enters in a list
and use the {\tt max()} and {\tt min()} functions to 
compute the maximum and minimum numbers after the 
loop completes.

\beforeverb
\begin{verbatim}
Enter a number: 6
Enter a number: 2
Enter a number: 9
Enter a number: 3
Enter a number: 5
Enter a number: done
Maximum: 9.0
Minimum: 2.0
\end{verbatim}
\afterverb
%

\end{ex}

